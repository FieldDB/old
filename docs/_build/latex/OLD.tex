% Generated by Sphinx.
\def\sphinxdocclass{report}
\documentclass[letterpaper,10pt,english]{sphinxmanual}
\usepackage[utf8]{inputenc}
\DeclareUnicodeCharacter{00A0}{\nobreakspace}
\usepackage[T1]{fontenc}
\usepackage{babel}
\usepackage{times}
\usepackage[Bjarne]{fncychap}
\usepackage{longtable}
\usepackage{sphinx}
\usepackage{multirow}


\title{OLD Documentation}
\date{February 16, 2013}
\release{1.0.0}
\author{Joel Dunham}
\newcommand{\sphinxlogo}{}
\renewcommand{\releasename}{Release}
\makeindex

\makeatletter
\def\PYG@reset{\let\PYG@it=\relax \let\PYG@bf=\relax%
    \let\PYG@ul=\relax \let\PYG@tc=\relax%
    \let\PYG@bc=\relax \let\PYG@ff=\relax}
\def\PYG@tok#1{\csname PYG@tok@#1\endcsname}
\def\PYG@toks#1+{\ifx\relax#1\empty\else%
    \PYG@tok{#1}\expandafter\PYG@toks\fi}
\def\PYG@do#1{\PYG@bc{\PYG@tc{\PYG@ul{%
    \PYG@it{\PYG@bf{\PYG@ff{#1}}}}}}}
\def\PYG#1#2{\PYG@reset\PYG@toks#1+\relax+\PYG@do{#2}}

\expandafter\def\csname PYG@tok@gd\endcsname{\def\PYG@tc##1{\textcolor[rgb]{0.63,0.00,0.00}{##1}}}
\expandafter\def\csname PYG@tok@gu\endcsname{\let\PYG@bf=\textbf\def\PYG@tc##1{\textcolor[rgb]{0.50,0.00,0.50}{##1}}}
\expandafter\def\csname PYG@tok@gt\endcsname{\def\PYG@tc##1{\textcolor[rgb]{0.00,0.25,0.82}{##1}}}
\expandafter\def\csname PYG@tok@gs\endcsname{\let\PYG@bf=\textbf}
\expandafter\def\csname PYG@tok@gr\endcsname{\def\PYG@tc##1{\textcolor[rgb]{1.00,0.00,0.00}{##1}}}
\expandafter\def\csname PYG@tok@cm\endcsname{\let\PYG@it=\textit\def\PYG@tc##1{\textcolor[rgb]{0.25,0.50,0.56}{##1}}}
\expandafter\def\csname PYG@tok@vg\endcsname{\def\PYG@tc##1{\textcolor[rgb]{0.73,0.38,0.84}{##1}}}
\expandafter\def\csname PYG@tok@m\endcsname{\def\PYG@tc##1{\textcolor[rgb]{0.13,0.50,0.31}{##1}}}
\expandafter\def\csname PYG@tok@mh\endcsname{\def\PYG@tc##1{\textcolor[rgb]{0.13,0.50,0.31}{##1}}}
\expandafter\def\csname PYG@tok@cs\endcsname{\def\PYG@tc##1{\textcolor[rgb]{0.25,0.50,0.56}{##1}}\def\PYG@bc##1{\setlength{\fboxsep}{0pt}\colorbox[rgb]{1.00,0.94,0.94}{\strut ##1}}}
\expandafter\def\csname PYG@tok@ge\endcsname{\let\PYG@it=\textit}
\expandafter\def\csname PYG@tok@vc\endcsname{\def\PYG@tc##1{\textcolor[rgb]{0.73,0.38,0.84}{##1}}}
\expandafter\def\csname PYG@tok@il\endcsname{\def\PYG@tc##1{\textcolor[rgb]{0.13,0.50,0.31}{##1}}}
\expandafter\def\csname PYG@tok@go\endcsname{\def\PYG@tc##1{\textcolor[rgb]{0.19,0.19,0.19}{##1}}}
\expandafter\def\csname PYG@tok@cp\endcsname{\def\PYG@tc##1{\textcolor[rgb]{0.00,0.44,0.13}{##1}}}
\expandafter\def\csname PYG@tok@gi\endcsname{\def\PYG@tc##1{\textcolor[rgb]{0.00,0.63,0.00}{##1}}}
\expandafter\def\csname PYG@tok@gh\endcsname{\let\PYG@bf=\textbf\def\PYG@tc##1{\textcolor[rgb]{0.00,0.00,0.50}{##1}}}
\expandafter\def\csname PYG@tok@ni\endcsname{\let\PYG@bf=\textbf\def\PYG@tc##1{\textcolor[rgb]{0.84,0.33,0.22}{##1}}}
\expandafter\def\csname PYG@tok@nl\endcsname{\let\PYG@bf=\textbf\def\PYG@tc##1{\textcolor[rgb]{0.00,0.13,0.44}{##1}}}
\expandafter\def\csname PYG@tok@nn\endcsname{\let\PYG@bf=\textbf\def\PYG@tc##1{\textcolor[rgb]{0.05,0.52,0.71}{##1}}}
\expandafter\def\csname PYG@tok@no\endcsname{\def\PYG@tc##1{\textcolor[rgb]{0.38,0.68,0.84}{##1}}}
\expandafter\def\csname PYG@tok@na\endcsname{\def\PYG@tc##1{\textcolor[rgb]{0.25,0.44,0.63}{##1}}}
\expandafter\def\csname PYG@tok@nb\endcsname{\def\PYG@tc##1{\textcolor[rgb]{0.00,0.44,0.13}{##1}}}
\expandafter\def\csname PYG@tok@nc\endcsname{\let\PYG@bf=\textbf\def\PYG@tc##1{\textcolor[rgb]{0.05,0.52,0.71}{##1}}}
\expandafter\def\csname PYG@tok@nd\endcsname{\let\PYG@bf=\textbf\def\PYG@tc##1{\textcolor[rgb]{0.33,0.33,0.33}{##1}}}
\expandafter\def\csname PYG@tok@ne\endcsname{\def\PYG@tc##1{\textcolor[rgb]{0.00,0.44,0.13}{##1}}}
\expandafter\def\csname PYG@tok@nf\endcsname{\def\PYG@tc##1{\textcolor[rgb]{0.02,0.16,0.49}{##1}}}
\expandafter\def\csname PYG@tok@si\endcsname{\let\PYG@it=\textit\def\PYG@tc##1{\textcolor[rgb]{0.44,0.63,0.82}{##1}}}
\expandafter\def\csname PYG@tok@s2\endcsname{\def\PYG@tc##1{\textcolor[rgb]{0.25,0.44,0.63}{##1}}}
\expandafter\def\csname PYG@tok@vi\endcsname{\def\PYG@tc##1{\textcolor[rgb]{0.73,0.38,0.84}{##1}}}
\expandafter\def\csname PYG@tok@nt\endcsname{\let\PYG@bf=\textbf\def\PYG@tc##1{\textcolor[rgb]{0.02,0.16,0.45}{##1}}}
\expandafter\def\csname PYG@tok@nv\endcsname{\def\PYG@tc##1{\textcolor[rgb]{0.73,0.38,0.84}{##1}}}
\expandafter\def\csname PYG@tok@s1\endcsname{\def\PYG@tc##1{\textcolor[rgb]{0.25,0.44,0.63}{##1}}}
\expandafter\def\csname PYG@tok@gp\endcsname{\let\PYG@bf=\textbf\def\PYG@tc##1{\textcolor[rgb]{0.78,0.36,0.04}{##1}}}
\expandafter\def\csname PYG@tok@sh\endcsname{\def\PYG@tc##1{\textcolor[rgb]{0.25,0.44,0.63}{##1}}}
\expandafter\def\csname PYG@tok@ow\endcsname{\let\PYG@bf=\textbf\def\PYG@tc##1{\textcolor[rgb]{0.00,0.44,0.13}{##1}}}
\expandafter\def\csname PYG@tok@sx\endcsname{\def\PYG@tc##1{\textcolor[rgb]{0.78,0.36,0.04}{##1}}}
\expandafter\def\csname PYG@tok@bp\endcsname{\def\PYG@tc##1{\textcolor[rgb]{0.00,0.44,0.13}{##1}}}
\expandafter\def\csname PYG@tok@c1\endcsname{\let\PYG@it=\textit\def\PYG@tc##1{\textcolor[rgb]{0.25,0.50,0.56}{##1}}}
\expandafter\def\csname PYG@tok@kc\endcsname{\let\PYG@bf=\textbf\def\PYG@tc##1{\textcolor[rgb]{0.00,0.44,0.13}{##1}}}
\expandafter\def\csname PYG@tok@c\endcsname{\let\PYG@it=\textit\def\PYG@tc##1{\textcolor[rgb]{0.25,0.50,0.56}{##1}}}
\expandafter\def\csname PYG@tok@mf\endcsname{\def\PYG@tc##1{\textcolor[rgb]{0.13,0.50,0.31}{##1}}}
\expandafter\def\csname PYG@tok@err\endcsname{\def\PYG@bc##1{\setlength{\fboxsep}{0pt}\fcolorbox[rgb]{1.00,0.00,0.00}{1,1,1}{\strut ##1}}}
\expandafter\def\csname PYG@tok@kd\endcsname{\let\PYG@bf=\textbf\def\PYG@tc##1{\textcolor[rgb]{0.00,0.44,0.13}{##1}}}
\expandafter\def\csname PYG@tok@ss\endcsname{\def\PYG@tc##1{\textcolor[rgb]{0.32,0.47,0.09}{##1}}}
\expandafter\def\csname PYG@tok@sr\endcsname{\def\PYG@tc##1{\textcolor[rgb]{0.14,0.33,0.53}{##1}}}
\expandafter\def\csname PYG@tok@mo\endcsname{\def\PYG@tc##1{\textcolor[rgb]{0.13,0.50,0.31}{##1}}}
\expandafter\def\csname PYG@tok@mi\endcsname{\def\PYG@tc##1{\textcolor[rgb]{0.13,0.50,0.31}{##1}}}
\expandafter\def\csname PYG@tok@kn\endcsname{\let\PYG@bf=\textbf\def\PYG@tc##1{\textcolor[rgb]{0.00,0.44,0.13}{##1}}}
\expandafter\def\csname PYG@tok@o\endcsname{\def\PYG@tc##1{\textcolor[rgb]{0.40,0.40,0.40}{##1}}}
\expandafter\def\csname PYG@tok@kr\endcsname{\let\PYG@bf=\textbf\def\PYG@tc##1{\textcolor[rgb]{0.00,0.44,0.13}{##1}}}
\expandafter\def\csname PYG@tok@s\endcsname{\def\PYG@tc##1{\textcolor[rgb]{0.25,0.44,0.63}{##1}}}
\expandafter\def\csname PYG@tok@kp\endcsname{\def\PYG@tc##1{\textcolor[rgb]{0.00,0.44,0.13}{##1}}}
\expandafter\def\csname PYG@tok@w\endcsname{\def\PYG@tc##1{\textcolor[rgb]{0.73,0.73,0.73}{##1}}}
\expandafter\def\csname PYG@tok@kt\endcsname{\def\PYG@tc##1{\textcolor[rgb]{0.56,0.13,0.00}{##1}}}
\expandafter\def\csname PYG@tok@sc\endcsname{\def\PYG@tc##1{\textcolor[rgb]{0.25,0.44,0.63}{##1}}}
\expandafter\def\csname PYG@tok@sb\endcsname{\def\PYG@tc##1{\textcolor[rgb]{0.25,0.44,0.63}{##1}}}
\expandafter\def\csname PYG@tok@k\endcsname{\let\PYG@bf=\textbf\def\PYG@tc##1{\textcolor[rgb]{0.00,0.44,0.13}{##1}}}
\expandafter\def\csname PYG@tok@se\endcsname{\let\PYG@bf=\textbf\def\PYG@tc##1{\textcolor[rgb]{0.25,0.44,0.63}{##1}}}
\expandafter\def\csname PYG@tok@sd\endcsname{\let\PYG@it=\textit\def\PYG@tc##1{\textcolor[rgb]{0.25,0.44,0.63}{##1}}}

\def\PYGZbs{\char`\\}
\def\PYGZus{\char`\_}
\def\PYGZob{\char`\{}
\def\PYGZcb{\char`\}}
\def\PYGZca{\char`\^}
\def\PYGZam{\char`\&}
\def\PYGZlt{\char`\<}
\def\PYGZgt{\char`\>}
\def\PYGZsh{\char`\#}
\def\PYGZpc{\char`\%}
\def\PYGZdl{\char`\$}
\def\PYGZti{\char`\~}
% for compatibility with earlier versions
\def\PYGZat{@}
\def\PYGZlb{[}
\def\PYGZrb{]}
\makeatother

\begin{document}

\maketitle
\tableofcontents
\phantomsection\label{index::doc}


Contents:
\setbox0\vbox{
\begin{minipage}{0.95\linewidth}
\textbf{Contents}

\medskip

\begin{itemize}
\item {} 
{\hyperref[user_guide:old-user-guide]{1   OLD User Guide}}

\item {} 
{\hyperref[user_guide:about-the-old]{2   About the OLD}}
\begin{itemize}
\item {} 
{\hyperref[user_guide:what-the-old-will-allow-you-to]{2.1   What the OLD Will Allow You To:}}

\item {} 
{\hyperref[user_guide:technical-specifications]{2.2   Technical Specifications:}}

\end{itemize}

\item {} 
{\hyperref[user_guide:how-to-get-the-old]{3   How to Get the OLD}}

\item {} 
{\hyperref[user_guide:how-to-use-the-old]{4   How to Use the OLD}}
\begin{itemize}
\item {} 
{\hyperref[user_guide:basics]{4.1   Basics}}

\item {} 
{\hyperref[user_guide:user-interface]{4.2   User Interface}}

\end{itemize}

\item {} 
{\hyperref[user_guide:forms]{5   Forms}}
\begin{itemize}
\item {} 
{\hyperref[user_guide:form-data]{5.1   Form Data}}
\begin{itemize}
\item {} 
{\hyperref[user_guide:id]{5.1.1   ID}}

\item {} 
{\hyperref[user_guide:transcription]{5.1.2   transcription}}

\item {} 
{\hyperref[user_guide:grammaticality]{5.1.3   grammaticality}}

\item {} 
{\hyperref[user_guide:morpheme-break]{5.1.4   morpheme break}}

\item {} 
{\hyperref[user_guide:morpheme-gloss]{5.1.5   morpheme gloss}}

\item {} 
{\hyperref[user_guide:gloss]{5.1.6   gloss}}

\item {} 
{\hyperref[user_guide:gloss-grammaticality]{5.1.7   gloss grammaticality}}

\item {} 
{\hyperref[user_guide:general-comments]{5.1.8   general comments}}

\item {} 
{\hyperref[user_guide:speaker-comments]{5.1.9   speaker comments}}

\item {} 
{\hyperref[user_guide:elicitation-method]{5.1.10   elicitation method}}

\item {} 
{\hyperref[user_guide:keywords]{5.1.11   keywords}}

\item {} 
{\hyperref[user_guide:category]{5.1.12   category}}

\item {} 
{\hyperref[user_guide:category-string]{5.1.13   category string}}

\item {} 
{\hyperref[user_guide:speaker]{5.1.14   speaker}}

\item {} 
{\hyperref[user_guide:elicitor]{5.1.15   elicitor}}

\item {} 
{\hyperref[user_guide:enterer]{5.1.16   enterer}}

\item {} 
{\hyperref[user_guide:verifier]{5.1.17   verifier}}

\item {} 
{\hyperref[user_guide:source]{5.1.18   source}}

\item {} 
{\hyperref[user_guide:date-elicited]{5.1.19   date elicited}}

\item {} 
{\hyperref[user_guide:date-and-time-entered]{5.1.20   date and time entered}}

\item {} 
{\hyperref[user_guide:date-and-time-last-modified]{5.1.21   date and time last modified}}

\end{itemize}

\item {} 
{\hyperref[user_guide:adding-a-form]{5.2   Adding a Form}}

\item {} 
{\hyperref[user_guide:searching-forms]{5.3   Searching Forms}}
\begin{itemize}
\item {} 
{\hyperref[user_guide:search-expressions]{5.3.1   search expressions}}

\item {} 
{\hyperref[user_guide:order-by-expression]{5.3.2   order by expression}}

\item {} 
{\hyperref[user_guide:additional-search-filters]{5.3.3   additional search filters}}

\item {} 
{\hyperref[user_guide:previous-searches]{5.3.4   previous searches}}

\item {} 
{\hyperref[user_guide:as-a-phrase]{5.3.5   as a phrase}}

\item {} 
{\hyperref[user_guide:all-of-these]{5.3.6   all of these}}

\item {} 
{\hyperref[user_guide:any-of-these]{5.3.7   any of these}}

\item {} 
{\hyperref[user_guide:exactly]{5.3.8   exactly}}

\item {} 
{\hyperref[user_guide:as-a-reg-exp]{5.3.9   as a reg exp}}

\item {} 
{\hyperref[user_guide:searching-and-orthographies]{5.3.10   searching and orthographies}}

\end{itemize}

\item {} 
{\hyperref[user_guide:browsing-forms]{5.4   Browsing Forms}}

\item {} 
{\hyperref[user_guide:other-form-actions]{5.5   Other Form Actions}}
\begin{itemize}
\item {} 
{\hyperref[user_guide:update]{5.5.1   update}}

\item {} 
{\hyperref[user_guide:delete]{5.5.2   delete}}

\item {} 
{\hyperref[user_guide:history]{5.5.3   history}}

\item {} 
{\hyperref[user_guide:associate]{5.5.4   associate}}

\item {} 
{\hyperref[user_guide:remember]{5.5.5   remember}}

\item {} 
{\hyperref[user_guide:export]{5.5.6   export}}

\item {} 
{\hyperref[user_guide:export-all]{5.5.7   export all}}

\item {} 
{\hyperref[user_guide:remember-all]{5.5.8   remember all}}

\end{itemize}

\end{itemize}

\item {} 
{\hyperref[user_guide:files]{6   Files}}
\begin{itemize}
\item {} 
{\hyperref[user_guide:file-data]{6.1   File Data}}
\begin{itemize}
\item {} 
{\hyperref[user_guide:id1]{6.1.1   ID}}

\item {} 
{\hyperref[user_guide:name]{6.1.2   name}}

\item {} 
{\hyperref[user_guide:mime-type]{6.1.3   MIME type}}

\item {} 
{\hyperref[user_guide:size]{6.1.4   size}}

\item {} 
{\hyperref[user_guide:id2]{6.1.5   enterer}}

\item {} 
{\hyperref[user_guide:description]{6.1.6   description}}

\item {} 
{\hyperref[user_guide:id3]{6.1.7   speaker}}

\item {} 
{\hyperref[user_guide:id4]{6.1.8   elicitor}}

\item {} 
{\hyperref[user_guide:id5]{6.1.9   date elicited}}

\item {} 
{\hyperref[user_guide:utterance-type]{6.1.10   utterance type}}

\item {} 
{\hyperref[user_guide:id6]{6.1.11   date and time entered}}

\item {} 
{\hyperref[user_guide:date-and-time-modified]{6.1.12   date and time modified}}

\end{itemize}

\item {} 
{\hyperref[user_guide:adding-a-file]{6.2   Adding a File}}

\item {} 
{\hyperref[user_guide:searching-files]{6.3   Searching Files}}

\item {} 
{\hyperref[user_guide:browsing-files]{6.4   Browsing Files}}

\item {} 
{\hyperref[user_guide:other-file-actions]{6.5   Other File Actions}}
\begin{itemize}
\item {} 
{\hyperref[user_guide:id7]{6.5.1   update}}

\item {} 
{\hyperref[user_guide:id8]{6.5.2   delete}}

\item {} 
{\hyperref[user_guide:id9]{6.5.3   associate}}

\item {} 
{\hyperref[user_guide:id10]{6.5.4   export all}}

\end{itemize}

\end{itemize}

\item {} 
{\hyperref[user_guide:collections]{7   Collections}}
\begin{itemize}
\item {} 
{\hyperref[user_guide:collection-data]{7.1   Collection Data}}
\begin{itemize}
\item {} 
{\hyperref[user_guide:id11]{7.1.1   ID}}

\item {} 
{\hyperref[user_guide:title]{7.1.2   title}}

\item {} 
{\hyperref[user_guide:type]{7.1.3   type}}

\item {} 
{\hyperref[user_guide:id12]{7.1.4   description}}

\item {} 
{\hyperref[user_guide:id13]{7.1.5   speaker}}

\item {} 
{\hyperref[user_guide:id14]{7.1.6   elicitor}}

\item {} 
{\hyperref[user_guide:id15]{7.1.7   source}}

\item {} 
{\hyperref[user_guide:id16]{7.1.8   date elicited}}

\item {} 
{\hyperref[user_guide:content]{7.1.9   content}}

\item {} 
{\hyperref[user_guide:associated-files]{7.1.10   associated files}}

\item {} 
{\hyperref[user_guide:associated-forms]{7.1.11   associated forms}}

\end{itemize}

\item {} 
{\hyperref[user_guide:viewing-a-collection]{7.2   Viewing a Collection}}

\item {} 
{\hyperref[user_guide:adding-a-collection]{7.3   Adding a Collection}}

\item {} 
{\hyperref[user_guide:searching-collections]{7.4   Searching Collections}}

\item {} 
{\hyperref[user_guide:browsing-collections]{7.5   Browsing Collections}}

\item {} 
{\hyperref[user_guide:other-collection-actions]{7.6   Other Collection Actions}}
\begin{itemize}
\item {} 
{\hyperref[user_guide:id17]{7.6.1   update}}

\item {} 
{\hyperref[user_guide:id18]{7.6.2   delete}}

\item {} 
{\hyperref[user_guide:id19]{7.6.3   associate}}

\end{itemize}

\end{itemize}

\item {} 
{\hyperref[user_guide:people]{8   People}}
\begin{itemize}
\item {} 
{\hyperref[user_guide:speakers]{8.1   Speakers}}
\begin{itemize}
\item {} 
{\hyperref[user_guide:view-a-speaker]{8.1.1   view a speaker}}

\item {} 
{\hyperref[user_guide:add-a-speaker]{8.1.2   add a speaker}}

\item {} 
{\hyperref[user_guide:update-a-speaker]{8.1.3   update a speaker}}

\item {} 
{\hyperref[user_guide:delete-a-speaker]{8.1.4   delete a speaker}}

\end{itemize}

\item {} 
{\hyperref[user_guide:users]{8.2   Users}}
\begin{itemize}
\item {} 
{\hyperref[user_guide:administrators]{8.2.1   administrators}}

\item {} 
{\hyperref[user_guide:contributors]{8.2.2   contributors}}

\item {} 
{\hyperref[user_guide:viewers]{8.2.3   viewers}}

\item {} 
{\hyperref[user_guide:view-a-user]{8.2.4   view a user}}

\item {} 
{\hyperref[user_guide:add-a-user]{8.2.5   add a user}}

\item {} 
{\hyperref[user_guide:edit-a-user]{8.2.6   edit a user}}

\item {} 
{\hyperref[user_guide:delete-a-user]{8.2.7   delete a user}}

\item {} 
{\hyperref[user_guide:user-settings]{8.2.8   user settings}}

\end{itemize}

\end{itemize}

\item {} 
{\hyperref[user_guide:tags]{9   Tags}}
\begin{itemize}
\item {} 
{\hyperref[user_guide:id20]{9.1   Keywords}}
\begin{itemize}
\item {} 
{\hyperref[user_guide:add-a-keyword]{9.1.1   add a keyword}}

\item {} 
{\hyperref[user_guide:edit-a-keyword]{9.1.2   edit a keyword}}

\item {} 
{\hyperref[user_guide:delete-a-keyword]{9.1.3   delete a keyword}}

\end{itemize}

\item {} 
{\hyperref[user_guide:syntactic-categories]{9.2   Syntactic Categories}}
\begin{itemize}
\item {} 
{\hyperref[user_guide:add-a-syntactic-category]{9.2.1   add a syntactic category}}

\item {} 
{\hyperref[user_guide:edit-a-syntactic-category]{9.2.2   edit a syntactic category}}

\item {} 
{\hyperref[user_guide:delete-a-syntactic-category]{9.2.3   delete a syntactic category}}

\end{itemize}

\item {} 
{\hyperref[user_guide:elicitation-methods]{9.3   Elicitation Methods}}
\begin{itemize}
\item {} 
{\hyperref[user_guide:add-an-elicitation-method]{9.3.1   add an elicitation method}}

\item {} 
{\hyperref[user_guide:edit-an-elicitation-method]{9.3.2   edit an elicitation method}}

\item {} 
{\hyperref[user_guide:id21]{9.3.3   delete a syntactic category}}

\end{itemize}

\end{itemize}

\item {} 
{\hyperref[user_guide:sources]{10   Sources}}
\begin{itemize}
\item {} 
{\hyperref[user_guide:add-a-source]{10.1   Add a Source}}

\item {} 
{\hyperref[user_guide:edit-a-source]{10.2   Edit a Source}}

\item {} 
{\hyperref[user_guide:delete-a-source]{10.3   Delete a Source}}

\end{itemize}

\item {} 
{\hyperref[user_guide:memory]{11   Memory}}
\begin{itemize}
\item {} 
{\hyperref[user_guide:forget-all]{11.1   forget all}}

\item {} 
{\hyperref[user_guide:id22]{11.2   export all}}

\end{itemize}

\item {} 
{\hyperref[user_guide:dictionary]{12   Dictionary}}
\begin{itemize}
\item {} 
{\hyperref[user_guide:browse]{12.1   Browse}}
\begin{itemize}
\item {} 
{\hyperref[user_guide:object-language-to-metalanguage]{12.1.1   object language to metalanguage}}

\item {} 
{\hyperref[user_guide:metalanguage-to-object-language]{12.1.2   metalanguage to object language}}

\end{itemize}

\item {} 
{\hyperref[user_guide:search]{12.2   Search}}
\begin{itemize}
\item {} 
{\hyperref[user_guide:id23]{12.2.1   object language to metalanguage}}

\item {} 
{\hyperref[user_guide:id24]{12.2.2   metalanguage to object language}}

\end{itemize}

\end{itemize}

\item {} 
{\hyperref[user_guide:application-settings]{13   Application Settings}}
\begin{itemize}
\item {} 
{\hyperref[user_guide:object-language]{13.1   Object Language}}

\item {} 
{\hyperref[user_guide:storage-input-output]{13.2   Storage, Input \& Output}}

\item {} 
{\hyperref[user_guide:object-language-orthography-ies]{13.3   Object Language Orthography(/ies)}}

\item {} 
{\hyperref[user_guide:object-language-data]{13.4   Object Language Data}}
\begin{itemize}
\item {} 
{\hyperref[user_guide:pure-object-language-fields]{13.4.1   Pure Object Language Fields}}

\item {} 
{\hyperref[user_guide:optionally-object-language-fields]{13.4.2   Optionally Object Language Fields}}

\item {} 
{\hyperref[user_guide:object-language-tag-fields]{13.4.3   Object Language Tag Fields}}

\end{itemize}

\item {} 
{\hyperref[user_guide:testing-orthography-conversion]{13.5   Testing Orthography Conversion}}

\item {} 
{\hyperref[user_guide:metalanguage]{13.6   Metalanguage}}

\end{itemize}

\item {} 
{\hyperref[user_guide:character-encoding]{14   Character Encoding}}

\item {} 
{\hyperref[user_guide:fonts]{15   Fonts}}

\item {} 
{\hyperref[user_guide:regular-expressions]{16   Regular Expressions}}
\begin{itemize}
\item {} 
{\hyperref[user_guide:metacharacters]{16.1   Metacharacters}}

\item {} 
{\hyperref[user_guide:some-useful-regular-expressions]{16.2   Some Useful Regular Expressions}}

\end{itemize}

\item {} 
{\hyperref[user_guide:old-markup]{17   OLD Markup}}
\begin{itemize}
\item {} 
{\hyperref[user_guide:old-form-embed-markup]{17.1   OLD Form Embed Markup}}

\item {} 
{\hyperref[user_guide:old-entity-link-markup]{17.2   OLD Entity Link Markup}}

\item {} 
{\hyperref[user_guide:old-file-file-markup]{17.3   OLD File file Markup}}

\item {} 
{\hyperref[user_guide:old-image-embed-markup]{17.4   OLD Image Embed Markup}}

\item {} 
{\hyperref[user_guide:old-metalanguage-markup]{17.5   OLD Metalanguage Markup}}

\end{itemize}

\item {} 
{\hyperref[user_guide:restructuredtext]{18   reStructuredText}}

\item {} 
{\hyperref[user_guide:markdown]{19   Markdown}}

\item {} 
{\hyperref[user_guide:keyboard-shortcuts]{20   Keyboard Shortcuts}}

\item {} 
{\hyperref[user_guide:browser-support]{21   Browser Support}}

\item {} 
{\hyperref[user_guide:glossary-of-terms]{22   Glossary of Terms}}

\item {} 
{\hyperref[user_guide:future-features]{23   Future Features}}
\begin{itemize}
\item {} 
{\hyperref[user_guide:priorities]{23.1   Priorities}}

\item {} 
{\hyperref[user_guide:relatively-easy-projects]{23.2   Relatively Easy Projects}}

\item {} 
{\hyperref[user_guide:big-projects]{23.3   Big Projects}}

\end{itemize}

\end{itemize}
\end{minipage}}
\begin{center}\setlength{\fboxsep}{5pt}\shadowbox{\box0}\end{center}


\chapter{1   OLD User Guide}
\label{user_guide:old-user-guide}\label{user_guide::doc}\label{user_guide:welcome-to-old-s-documentation}
This guide provides general information about what the Online Linguistic
Database (OLD) is and how it should be used.


\chapter{2   About the OLD}
\label{user_guide:about-the-old}
The OLD is a web application designed to be used by a group of individuals in
order to document, analyze and learn a particular natural language.

As a linguistics PhD student whose research involves the documentation and
analysis of understudied languages, I was originally motivated to create the
OLD because I saw a lack of open source, cross-platform and multi-user database
applications suitable for documenting linguistic data.  The OLD is my effort to
fill that void.


\section{2.1   What the OLD Will Allow You To:}
\label{user_guide:what-the-old-will-allow-you-to}\begin{itemize}
\item {} 
build a collaborative and ever-growing online database of linguistic data on a
particular language

\item {} 
create a dictionary-like interface to your language data

\item {} 
organize your linguistic data into an intelligent structure

\item {} 
perform powerful searches on your data, utilizing regular expressions and
boolean operators with multiple restrictors on multiple fields

\item {} 
export data to a variety of formats (plain text, XML, LaTeX)

\item {} 
incorporate non-textual data types (audio/video/images as recordings or
stimuli) and specify the relationships between textual and non-textual data
types

\item {} 
document multi-sentence texts such as stories or records of elicitations

\item {} 
collaborate and share data with other researchers authorized to use your OLD
application

\item {} 
control access to your data via password-protected accounts for registered
users

\end{itemize}


\section{2.2   Technical Specifications:}
\label{user_guide:technical-specifications}\begin{itemize}
\item {} 
programming language is Python (2.6), using the Pylons (0.9.7) web framework

\item {} 
model is a relational database (usually MySQL or SQLite) abstracted by
SQLAlchemy

\item {} 
user interface is HTML, CSS and Javascript

\end{itemize}


\chapter{3   How to Get the OLD}
\label{user_guide:how-to-get-the-old}
The OLD is open source software licensed under the GPL.  That means you can
download it, use it and alter its source code, but you cannot sell it for profit.

The web page of the OLD is \href{http://www.onlinelinguisticdatabase.org}{www.onlinelinguisticdatabase.org} and the source code
is hosted on Google Code's Project Hosting
(\href{http://code.google.com/p/onlinelinguisticdatabase}{code.google.com/p/onlinelinguisticdatabase}).

In order to create an OLD application, you must download the OLD, install it
on a server and customize it to suit the requirements of your language.  It is
possible to install the OLD on, and run/test it from, your own computer without
the need for a server.

The OLD web page provides instructions on downloading the OLD and setting up a
language-specific OLD application.


\chapter{4   How to Use the OLD}
\label{user_guide:how-to-use-the-old}
This section describes how to use an OLD application, i.e., how to add,
structure, search for, export, and otherwise interact with language data.


\section{4.1   Basics}
\label{user_guide:basics}
At the heart of the OLD are three types of entity: Forms, Files and Collections.
\begin{figure}[htbp]
\centering
\capstart

\includegraphics{images/OLD_structure.jpg}
\caption{The three entities of the OLD}\end{figure}

A Form is a morpheme, word, phrase or sentence of the object language.  Forms
minimally consist of a transcription and a gloss (i.e., translation in the
metalanguage).  Optional data include a morphemic analysis, general comments,
date elicited, etc.  As the diagram indicates, a Form may be associated to
(i.e., reference or point to) one or more Files (see below).

A File is an audio, image, video, or text file.  A File could represent a
recording of an utterance, a depiction of a context, or a stimulus used in
elicitation.

A Collection is an ordered list of Forms.  A Collection might represent a story
or a record of an elicitation or any other multi-sentence (and therefore
multi-Form) piece of discourse.  Like a Form, a Collection can be associated
to one or more Files.

Interacting with an OLD application typically involves creating, searching for
or updating a Form, File or Collection.  Other common actions include adding or
modifying people (users or speakers), tags (keywords, syntactic categories or
elicitation methods) or sources.


\section{4.2   User Interface}
\label{user_guide:user-interface}
The primary menu contains the database, dictionary, help and settings options.

The contents of the secondary menu depend on the currently active primary menu
option.  When the database option is active in the primary menu, the secondary
menu contains the people, tags, sources and memory options.  When the dictionary
option is active, the secondary menu contains browse and search options.  The
help and settings primary menu options make the secondary menu disappear.

The side menu contains links to common actions on OLD entities, namely adding
and searching for Forms, Files and Collections.
\begin{figure}[htbp]
\centering
\capstart

\includegraphics{images/OLD_interface.png}
\caption{The basic OLD interface}\end{figure}


\chapter{5   Forms}
\label{user_guide:forms}
As stated above, a Form is a grouping of data that are about a morpheme, word,
phrase or sentence of the object language.


\section{5.1   Form Data}
\label{user_guide:form-data}
This subsection describes the types of data that comprise a Form.  In the
context of the relational database model, these types of data are the columns of
the Form table.


\subsection{5.1.1   ID}
\label{user_guide:id}
The ID is a unique integer assigned by the RDBMS to each Form upon creation.
Knowing the ID of a Form comes in handy when you want to associate that Form to
a Collection.  It is also good to know for when you want to quickly access a
Form.

For example, enter ``domain\_name/form/view/11'' (where
``domain\_name'' is your OLD application's domain name, e.g., ``www.old.org'') in
your browser's address bar to view form with ID 11.

You can also enter a comma-separated list of Form IDs to quickly access several
Forms: ``domain\_name/form/view/11,12,13,14''


\subsection{5.1.2   transcription}
\label{user_guide:transcription}
The transcription is a textual representation of the sound of a Form.

A transcription is an object language string and as such it should be written
using the graphs of the input orthography.  (If not specified in the current
user's settings, the input orthography is the default input orthography
specified in the application settings.)  Before being stored in the database,
the transcription will be converted from the input orthography to the
storage orthography.

The recommended position of the transcription along the ``broad-narrow''
``phonetic/phonemic'' spectrum should be stated in your particular OLD
application's help section.

A multi-sentence discourse should not be entered as a single form, but as
multiple Forms, grouped together and ordered into a Collection.

It is recommended that transcriptions contain only the following (strings of)
characters:
\begin{itemize}
\item {} 
graphs from the input orthography

\item {} 
standard punctuation: '' ` ( ) ! ? , ; :

\end{itemize}

(In future versions of the OLD, functionality may be created that
would output a warning after, or even disallow, entry of transcriptions
violating these conditions.)


\subsection{5.1.3   grammaticality}
\label{user_guide:grammaticality}
The grammaticality indicates whether a particular object language Form is
grammatical (), ungrammatical (`*') or questionable (`?').  (Either more
grammaticality options should be added as standard or
administrators/contributors should be able to customize the forced-choice
grammaticality field.)

When adding a Form, grammaticality judgments should not be indicated in the text
of the transcription.  Instead, use the forced choice select field to the left
of the transcription input field.


\subsection{5.1.4   morpheme break}
\label{user_guide:morpheme-break}
The morpheme break field contains a morphological analysis of the form.  The
system is currently preset to expect `-` and `=' as morpheme delimiters and ` `
as the word delimiter, but this could/should be made customizable on an
application-specific basis.

The morpheme break field may or may not be specified as an object language
string field.  Such specification is made by administrators in the application
settings page.  If the morpheme break field is set up as an object language
string field, then morpheme break input will be converted to the storage
orthography for storage and converted to the output orthography for display
(just like the data in the transcription field.)  If the morpheme break field is
not set up as an object language string field, then no conversion will be
applied.

Whenever a Form is entered or updated, the OLD attempts to identify all of that
Form's morpheme-gloss pairs and searches for matches in existing Forms.  If one
or more matches are found, then the morpheme and its gloss are displayed as HTML
links the match(es).  This allows users to immediately see the extent to which
their morphological analyses are consistent with the rest of the data in the
system.

To understand this morphological linking in detail, imagine an OLD application
containing the following two Forms:

\begin{tabulary}{\linewidth}{|L|L|}
\hline

ID
 & 
1
\\\hline

transcription
 & 
chien
\\\hline

morpheme break
 & 
chien
\\\hline

morpheme gloss
 & 
dog
\\\hline

gloss
 & 
dog, mutt
\\\hline
\end{tabulary}


\begin{tabulary}{\linewidth}{|L|L|}
\hline

ID
 & 
2
\\\hline

transcription
 & 
s
\\\hline

morpheme break
 & 
s
\\\hline

morpheme gloss
 & 
plrl
\\\hline

gloss
 & 
plural marker
\\\hline
\end{tabulary}


Now, when the following Form is entered,

\begin{tabulary}{\linewidth}{|L|L|}
\hline

transcription
 & 
chiens
\\\hline

morpheme break
 & 
chien-s
\\\hline

morpheme gloss
 & 
dog-PL
\\\hline

gloss
 & 
dogs
\\\hline
\end{tabulary}


the system identifies the following morpheme-gloss pairs (`chien'-`dog' and
`s'-`PL').  It first searches the database for a Form with `chien' as its
morpheme break value and `dog' as its morpheme gloss value.  It finds such a
match in Form 1 and as a result it displays both `chien' and `dog' in our newly
entered Form as links to Form 1.  The link is displayed in blue font to indicate
a perfect match.

Its second search is for a Form with `s' as its morpheme break value and `PL' as
its morpheme gloss value.  A match is found in Form 2, but it is partial because
the morpheme gloss value of Form 2 is `plrl' and not `PL'.  Therefore, `s' in
our new Form will be displayed as a green link (green to indicate a partial
match) to Form 2 and `PL' will not be displayed as a link.


\subsection{5.1.5   morpheme gloss}
\label{user_guide:morpheme-gloss}
The morpheme gloss field should contain a gloss in the metalanguage for each
object language morpheme listed in the morpheme break field.  The same
delimiters should be used between the morpheme glosses as were used between the
morphemes in the morpheme break line.

Researchers of an OLD application might want to work toward a consensus on how
morphemes should be glossed.

Morpheme glosses will be displayed as links to matching Forms in the manner
described above in the section on morpheme breaks.


\subsection{5.1.6   gloss}
\label{user_guide:gloss}
The gloss is a translation of the Form into the metalanguage.  When the Form
represents a spatio-temporally located utterance, whenever possible the gloss
should be something that the speaker offered, or would at least consent to, as a
translation.

The OLD allows multiple (up to four) glosses for a single Form.  Each gloss has
its own gloss grammaticality field.  This makes it possible to document a Form
as compatible with certain glosses but not with others.  For example, a form
about the French word `banque' might have `bank (financial institutition)' as
its first gloss and `*riverbank' as its second gloss.


\subsection{5.1.7   gloss grammaticality}
\label{user_guide:gloss-grammaticality}
As discussed in the gloss section above, each gloss may have its own
grammaticality.  This grammaticality indicates the acceptability of the Form
with a particular translation into the metalanguage.  At present, the OLD allows
three choices: compatible (`'), incompatible (`*') and questionable (`?').


\subsection{5.1.8   general comments}
\label{user_guide:general-comments}
The general comments field is intended to contain notes pertaining to the Form
in question.  If you don't know where else to document something about a Form,
enter it in the general comments field.


\subsection{5.1.9   speaker comments}
\label{user_guide:speaker-comments}
The speaker comments field is intended to contain quotations (or paraphrases)
from the speaker of a particular Form.  Often a comment from the speaker is
not appropriate as a gloss yet it contains valuable information about some
aspect of the Form.


\subsection{5.1.10   elicitation method}
\label{user_guide:elicitation-method}
The elicitation method refers to the means via which a particular Form was
obtained.  Often, for example, it is useful to know whether the speaker
translated a metalanguage utterance of the elicitor, or described a visually
represented context or judged the grammaticality of an object language utterance
made by the elicitor, or whether the Form was obtained in some other manner.

The elicitation method field is a forced-choice user-populated field.  That is,
researchers must choose from a list of possible elicitation methods, but that
list can be modified by researchers.  By default there are no elicitation
methods predefined by an OLD application.  Users must click on ``database'' in the
primary menu and then ``tags'' in the secondary menu in order to add (or possibly
update) the list of elicitation methods.  (See the elicitation methods section).
The intention behind forced-choice user-populated fields is to encourage intra-
and inter-user consistency.


\subsection{5.1.11   keywords}
\label{user_guide:keywords}
Keywords provide users with a general-purpose way of tagging Forms.  A single
Form may be associated to zero, one or many keywords.  Keywords are defined by
users of the OLD application in question.  Click on ``database'' in the primary
menu and then ``tags'' in the secondary menu to add new keywords.


\subsection{5.1.12   category}
\label{user_guide:category}
The category refers to the syntactic or morphological category of the Form.
Like elicitation method, category is a forced-choice user-populated field which
is initially empty in a new OLD application.  Researchers can add new categories
(e.g., S, N, V, A, Adv, etc.) by clicking on ``database'' in the primary menu and
``tags'' in the secondary menu.  (See the category section.)


\subsection{5.1.13   category string}
\label{user_guide:category-string}
The category string is a string representing the morpho-syntactic categories of
the morphemes within the Form.  This string is generated by the system based on
the morpheme break and morpheme gloss data entered by the user.  For example,
suppose that the following form has just been entered.

\begin{tabulary}{\linewidth}{|L|L|}
\hline

transcription
 & 
chiens
\\\hline

morpheme break
 & 
chien-s
\\\hline

morpheme gloss
 & 
dog-PL
\\\hline

gloss
 & 
dogs
\\\hline
\end{tabulary}


Suppose further that when the system searches for the morpheme-gloss pairs
`chien'-`dog' and `s'-`PL' it finds and exact match for each.  In that scenario,
the categories of the `chien'-`dog' and `s'-`PL' Forms (lets say they are `N'
and `Agr') will be used to generate the category string of `chiens' and the
result will be `N-Agr'.

When an OLD application contains many Forms whose morpheme break and morpheme
gloss fields are consistent with the system's own lexical Forms, many category
strings will be generated.  When this is the case, users can search the category
strings to reveal high-level morpho-syntactic patterns.


\subsection{5.1.14   speaker}
\label{user_guide:speaker}
The speaker is the individual who uttered the object language token that the
Form represents.  To view the list of speakers documented in an OLD application,
click on ``database'' in the primary menu and ``people'' in the secondary menu.
Both administrators and contributors may add new speakers to the system (see the
speaker section).


\subsection{5.1.15   elicitor}
\label{user_guide:elicitor}
The elicitor is the researcher who elicited the Form, that is, the person who
recorded and/or transcribed the utterance of a speaker.  Entering an elicitor
involves choosing from a list of people registered as researchers for the OLD
application in question.  To view the list of registered researchers of an OLD
application, click on ``database'' in the primary menu and ``people'' in the
secondary menu.


\subsection{5.1.16   enterer}
\label{user_guide:enterer}
The enterer field is automatically populated with the name of the OLD researcher
who is adding the Form.  In order to add a Form, a person must be logged in to
the OLD application.


\subsection{5.1.17   verifier}
\label{user_guide:verifier}
The verifier of a Form is another (perhaps more experienced) researcher who has
already elicited a near-identical utterance and wants to indicate her agreement
about the accuracy of the first researcher's documentation of that utterance.
(The verifier field might be seldom used in practice...)


\subsection{5.1.18   source}
\label{user_guide:source}
This category refers to the textual source of a Form, if applicable.
Researchers can add new sources by clicking on ``database'' in the primary menu
and ``sources'' in the secondary menu.


\subsection{5.1.19   date elicited}
\label{user_guide:date-elicited}
The date when the Form was elicited (if applicable) is documented in the date
elicited field in mm/dd/yyyy format.


\subsection{5.1.20   date and time entered}
\label{user_guide:date-and-time-entered}
The date and time when the Form was entered into the OLD application is
automatically generated upon entry by the system.


\subsection{5.1.21   date and time last modified}
\label{user_guide:date-and-time-last-modified}
The date and time when the Form was last updated (modified) is automatically
generated by the system during each update.


\section{5.2   Adding a Form}
\label{user_guide:adding-a-form}
To add a Form, click on ``add'' under ``forms'' in the side menu or use the keyboard
shortcut ``a''.

Users can press the tab button to focus each of the fields in top-to-bottom,
left-to-right order.  If more than one gloss needs to be added, click on the ``+''
button to the right of the first visible gloss field.  The only mandatory fields
are the transcription and gloss fields.  Once all necessary data has been
entered, click the ``Add Form'' button at the bottom of the page.

Some general guidelines for entering Forms:
\begin{itemize}
\item {} 
in the transcription field, only use the characters (or character sequences)
specified in the input orthography (plus standard punctuation).  Remember,
a user can specify her own input orthography and said orthography may differ
from the system-wide default input orthography.

\item {} 
in the morpheme break field, only use the characters (or character sequences)
specified in the input orthography of your OLD application plus standard
punctuation and morpheme delimiters.  (This assumes your OLD application is
treating morpheme break data as strings of the object language; see the XXX
section for details.)

\item {} 
try to be consistent in your transcriptions, your spelling of morphemes and
your spelling of glosses, if not with the practice of other users, then at
least with your own past practice; such consistency will help you (and others)
to find quickly the data you need

\item {} 
if you forget how a morpheme or gloss has been transcribed, open a Form search
page in a new browser window/tab and search for your morpheme/gloss to find
out past spellings (hint: it is often useful to have serveral OLD browser tabs
open at once)

\end{itemize}

What to do if the database already contains a Form very similar to the one you
are about to add:
\begin{itemize}
\item {} 
if your Form represents a recording of an utterance event, it is usually best
to enter your Form despite the quasi-duplication it will cause

\item {} 
if your Form represents an abstraction (e.g., a morpheme or word), then it is
usually best to not enter a duplicate Form

\end{itemize}


\section{5.3   Searching Forms}
\label{user_guide:searching-forms}
The OLD allows you to perform powerful searches on your Form data.  The
screenshot below shows the OLD Form search page.
\begin{figure}[htbp]
\centering
\capstart

\includegraphics{images/OLD_form_search.png}
\caption{The standard OLD Form search page}\end{figure}


\subsection{5.3.1   search expressions}
\label{user_guide:search-expressions}
Users can enter one or two search expressions and these expressions can be
coordinated via conjunction (`and'), disjunction (`or') or negated conjunction
(`and not').

Each search expressions is comprised of (i) a text input field (ii) a search
type select field and (iii) a search location select field.

The text input field is where one enters the pattern that the result data must
match.

The search type select field indicates the way in which the search is to
be implemented.  The search type options are `as a phrase', `all of these',
`any of these', `as a reg exp' and `exactly'.  These will be discussed in more
detail below.

The search location select field lists options for where (i.e., which column of
the Form table) the system should look to match the pattern.  The search
location options are `transcription', `gloss', `morpheme break', `morpheme
gloss', `general comments', `speaker comments', `syntactic category string' and
`ID'.


\subsection{5.3.2   order by expression}
\label{user_guide:order-by-expression}
The order by expression allows one to state the order in which the matching
results should be returned.  One thing to note about the order by expression is
that it will (probably) not order your results according to the order of graphs
in the orthography specified in your OLD application settings.  The order by
expression uses the code points (in utf-8 encoding) of the characters in order
to determine order.  If your orthography contains characters outside of the 26
standard English ones, then the code points of those characters will likely be
quite high and those characters will be understood by the system as coming after
`z'.


\subsection{5.3.3   additional search filters}
\label{user_guide:additional-search-filters}
Additional search filters can be added to your search by clicking on the `+'
button next to `additional search filters'.  Here one can further refine a
Form search by putting conditions on the speaker, elicitor, enterer, verifier,
source, grammaticality, gloss grammaticality, elicitation method, (syntactic)
category, keywords, date elicited, date entered and/or date modified.


\subsection{5.3.4   previous searches}
\label{user_guide:previous-searches}
One can repeat or make modifications to a previous search by clicking
on the ``previous searches'' button at the bottom of the Form search page.  The
OLD keeps a record of each user's last ten searches and displays them when this
button is clicked.  When a past search is clicked, the search Form is returned
with the appropriate fields set so that clicking ``Search Forms'' will repeat the
search.  This functionality is good for repeating searches as well as for making
modifications to previous searches without having to re-enter all the search
criteria.

The following subsections describe how to use each of the different search types
available in a search expression.


\subsection{5.3.5   as a phrase}
\label{user_guide:as-a-phrase}
The `as a phrase' search type option causes the system to return all forms where
the search location contains the specified search term as a substring.  Thus,
searching for the pattern `the' as a phrase in the transcription field will
return all Forms where the string `the' is in the transcription, e.g., `the
dog', `they left', `another thing', etc.


\subsection{5.3.6   all of these}
\label{user_guide:all-of-these}
The `all of these' search type option causes the search pattern to be split by
whitespace into sub-patterns and returns all Forms where the chosen location
contains all of the sub-patterns.  For example, searching for `the and' with the
`all of these' search type option in the transcription field will return all
Forms where both `the' and `and' are in the transcription, e.g., `the sand',
`the cat and the dog', `Mandy hit her brother', etc.


\subsection{5.3.7   any of these}
\label{user_guide:any-of-these}
The `any of these' search type option causes the search pattern to be split by
whitespace into sub-patterns and returns all Forms where the chosen location
contains any of the sub-patterns.  For example, searching for `the and' with the
`any of these' search type option in the transcription field will return all
Forms where either `the' and `and' are in the transcription, e.g., `the dog',
`Mandy ran', `John and Mary smiled', `other people', etc.


\subsection{5.3.8   exactly}
\label{user_guide:exactly}
The `exactly' search type option returns all Forms where the selected search
location contains nothing but the search pattern.  For example, searching for
`dog' in the transcription with `exactly' as the search type option will return
all Forms where the transcription value contains the string `dog' and nothing
else.


\subsection{5.3.9   as a reg exp}
\label{user_guide:as-a-reg-exp}
The `as a reg exp' search type option causes the search term to be interpreted
as a regular expression.  Regular expressions use a certain syntax which allows
you to specify complex patterns.  For example, using regular expressions one
could search for the word `the' and avoid matching `other' or `they' or one
could search for strings that begin with `t'.  Regular expression searches are
very powerful but require learning a bit of the regular expression syntax.  See
the {\hyperref[user_guide:regular-expressions]{Regular Expressions}} section for more details.


\subsection{5.3.10   searching and orthographies}
\label{user_guide:searching-and-orthographies}
The text input of a search expression whose location is transcription (or
morpheme break, depending on the system settings) will be converted from the
input orthography to the storage orthography before the query is performed.


\section{5.4   Browsing Forms}
\label{user_guide:browsing-forms}
There are two ways to browse all forms:
\begin{enumerate}
\item {} 
Click on ``forms'' in the side menu.  This will return all Forms in the
database, ordered by transcription ascending.

\item {} 
Enter an empty search, i.e., just go to the Form search page and hit enter
or click on ``Search Forms''.  This method allows you to change the default
ordering

\end{enumerate}

(You can also browse forms via the dictionary interface to the database (see the
{\hyperref[user_guide:dictionary]{Dictionary}} section).  However, the dictionary lists only those Forms whose
transcriptions contain no white space.)


\section{5.5   Other Form Actions}
\label{user_guide:other-form-actions}
Beneath each Form are six buttons used to perform actions related to that Form:
`update', `delete', `history', `associate', `remember' and `export'.
\begin{figure}[htbp]
\centering
\capstart

\includegraphics{images/OLD_form_example.png}
\caption{An example Form showing the six action buttons}\end{figure}


\subsection{5.5.1   update}
\label{user_guide:update}
Clicking on the update button brings up the update Form page which looks almost
identical to the add Form page except that the fields are populated with values.

Before a Form is updated, it is first saved in a form\_backup table of the
database.  This means that all versions of a Form are recorded.  The history
button (see below) allows one to view the previous versions of a Form.


\subsection{5.5.2   delete}
\label{user_guide:delete}
The delete button deletes a Form, i.e., removes it from the form table of the
database.  If one clicks the delete button, a confirm dialog appears in order to
ensure that Forms are not mistakenly deleted.

As with the update action, before a Form is deleted it is saved in the
form\_backup table of the database.  This means that a mistakenly deleted Form
can be restored.  (But this requires advanced knowledge of the system...)


\subsection{5.5.3   history}
\label{user_guide:history}
The history button reveals the Form followed by its previous versions, ordered
by newest to oldest.  Thus, if a Form has been changed, one can see how it used
to be and change it back if desired.

(It might be useful to have ``revert'' buttons next to each previous version.  It
might also be useful to use string comparisons and highlighting to illustrate
the exact nature of the changes.)


\subsection{5.5.4   associate}
\label{user_guide:associate}
The associate button directs to the associate Form page which allows one to
associate the Form to one or more Files by entering one or more comma-separated
File IDs in a text input field.  Associated Files appear beneath the Form
data when the Form is displayed.


\subsection{5.5.5   remember}
\label{user_guide:remember}
Clicking the remember button causes the Form to be remembered in the current
user's Memory.  Memory is simply a user-specific ordered list of references to
Forms.  See the {\hyperref[user_guide:memory]{Memory}} section.


\subsection{5.5.6   export}
\label{user_guide:export}
The export button brings up the export options page.  This page provides a
number of export types.  At the time of writing, there are these 7:
\begin{enumerate}
\item {} 
transcription only (.txt)

\item {} 
transcription and gloss

\item {} 
interlinear gloss text

\item {} 
interlinear gloss text +

\item {} 
tab-delimited: everything

\item {} 
XeLaTeX IGT (Covington)

\item {} 
XeLaTeX IGT (Covington) +

\item {} 
XeLaTeX IGT (Covington) ++

\end{enumerate}

Choosing an option and clicking ``Export'' will generate an export file and return
the export complete page with a link for downloading your generated export file.

New export options can be created with a minimal amount of coding.  Let the
developer know what kind of export file you want to generate and he may try to
implement it.  (Better yet, study the lib/exporter.py module and write your
own exporter object based on the examples therein -- and don't forget to share.)


\subsection{5.5.7   export all}
\label{user_guide:export-all}
When the results of a Form search are displayed, an `export all' button appears
in the upper left portion of the page.  Clicking this button brings up the
export page where one can choose the export type (see above).  Export all works
just like export except that in this case the output can contain data from more
than one Form.


\subsection{5.5.8   remember all}
\label{user_guide:remember-all}
When the results of a Form search are displayed, a `remember all' button appears
in the upper left portion of the page.  Clicking this button will put all the
search results into the current user's memory (see {\hyperref[user_guide:memory]{Memory}})


\chapter{6   Files}
\label{user_guide:files}
An OLD File is a grouping of data about a digital file.  The file is uploaded to
an OLD application and data about the file (such as its name, type and size) are
inferred in the process.  Other data (such as a description) may be entered by
the user.


\section{6.1   File Data}
\label{user_guide:file-data}
This section describes the type of data that comprise an OLD File.


\subsection{6.1.1   ID}
\label{user_guide:id1}
The ID of a File is a unique integer automatically generated by the database.
Knowing the ID of a File can be useful when you want to search for it, associate
it to a Form or Collection or when you want to quickly view a certain file (just
enter enter ``domain\_name/file/view/1'' in your browser's address bar to view the
File with ID 1).


\subsection{6.1.2   name}
\label{user_guide:name}
The name of the uploaded file becomes the name of the OLD File.  If the system
already contains a file with that name, a digit is appended to the file name and
incrememted until a unique name is arrived at.


\subsection{6.1.3   MIME type}
\label{user_guide:mime-type}
The MIME type of an OLD File is inferred from the a uploaded file.  The MIME
type is a string that indicates whether the file is text, an image, a video, a
sound file, a PDF, a Word document, etc.  For more details, see the \href{http://en.wikipedia.org/wiki/MIME\_type}{MIME type
Wikipedia entry}


\subsection{6.1.4   size}
\label{user_guide:size}
The size of the OLD File is the size of the uploaded file.


\subsection{6.1.5   enterer}
\label{user_guide:id2}
The enterer is automatically added as a reference to the user who entered the
File.


\subsection{6.1.6   description}
\label{user_guide:description}
The description is a general-purpose string of text entered by the user.  The
description might contain information about what happens on an audio/video
recording or what a certain audio/video/image stimulus was designed to elicit.


\subsection{6.1.7   speaker}
\label{user_guide:id3}
If a File is about a recording of a linguistic event, the speaker field can be
used to indicate who was speaking the object language during the recording.

(Functionality should probably exist to associate more than one speaker to a
File...)


\subsection{6.1.8   elicitor}
\label{user_guide:id4}
If a File is about a recording of a linguistic event, the elicitor field can be
used to indicate the researcher who was eliciting during the recording.


\subsection{6.1.9   date elicited}
\label{user_guide:id5}
If a File is about a recording of a linguistic event, the date when the
recording was made can be entered in the date elicited field in mm/dd/yyyy
format.


\subsection{6.1.10   utterance type}
\label{user_guide:utterance-type}
If a File is about a recording of a linguistic event, then one can choose one of
the following four options for utterance type:
\begin{enumerate}
\item {} 
None: the type of utterance is left unspecified (the default)

\item {} 
Object Language Utterance: the recording contains only utterances of the
object language

\item {} 
Metalanguage Utterance: the recording contains only utterances of the
metalangauge

\item {} 
Mixed Utterance: the recording contains utterances of both the object
language and metalanguage

\end{enumerate}

(Choosing the appropriate utterance type for a File representing a recording will
be useful for certain planned OLD features, e.g., language learning games that
might play a recording of an object language utterance and have players try to
guess the meaning.)


\subsection{6.1.11   date and time entered}
\label{user_guide:id6}
When a File is added, the system automatically records the date and time the
File was entered.


\subsection{6.1.12   date and time modified}
\label{user_guide:date-and-time-modified}
When a File is updated, the system automatically records the date and time in
the date and time modified field.


\section{6.2   Adding a File}
\label{user_guide:adding-a-file}
To add a File, go to `add' under `files' in the side menu (or use the keyboard
shortcut `q').  Adding a File requires that you upload a digital file to the OLD
application.  Just click on the `browse' button and choose a file from your file
system. After choosing the correct file, click the `Add File' button at
the bottom of the page.

One may also optionally add a description, specify a speaker, an elicitor, a
date elicited and/or an utterance type.

An OLD application will accept for upload digital files with the following MIME
types:
\begin{itemize}
\item {} 
text/plain (plain text)

\item {} 
application/x-latex (LaTeX document)

\item {} 
application/msword (MS Word document)

\item {} 
application/vnd.oasis.opendocument.text (Open Document Format)

\item {} 
application/pdf (PDF)

\item {} 
image/gif

\item {} 
image/jpeg

\item {} 
image/png

\item {} 
audio/mpeg

\item {} 
audio/ogg

\item {} 
audio/x-wav

\item {} 
video/mpeg

\item {} 
video/mp4

\item {} 
video/ogg

\item {} 
video/quicktime

\item {} 
video/x-ms-wmv

\end{itemize}


\section{6.3   Searching Files}
\label{user_guide:searching-files}
The File search page is very similar to the Form search page (see
{\hyperref[user_guide:searching-forms]{Searching Forms}}).  Note the following differences:
\begin{itemize}
\item {} 
Files and Forms have different search location options.  Currently one can
search Files by name, description and ID.

\item {} 
Files do not have previous searches functionality

\item {} 
Files have different additional search filters; note in particular the
MIME type (see {\hyperref[user_guide:mime-type]{MIME type}}) and size filters

\end{itemize}

To search for a File, click on `search' under `files' in the side menu or use
the keyboard shortcut `w'.


\section{6.4   Browsing Files}
\label{user_guide:browsing-files}
To browse all Files, click on `files' in the side menu or use the keyboard
shortcut `r'.  To browse all Files in a particular order, perform an empty File
search and alter the order by expression as appropriate.


\section{6.5   Other File Actions}
\label{user_guide:other-file-actions}
When a File is displayed, three action buttons appear at the bottom: `update',
`delete' and `associate'.
\begin{figure}[htbp]
\centering
\capstart

\includegraphics{images/OLD_file_example.png}
\caption{An example File showing the three action buttons}\end{figure}


\subsection{6.5.1   update}
\label{user_guide:id7}
Clicking on the update button displays the File update page which looks very
similar to the File add page except that some of the fields may contain data and
the uploaded digital file cannot be changed.  Make changes and click `Save
Changes'.


\subsection{6.5.2   delete}
\label{user_guide:id8}
The delete button will both delete the data about the File from the database as
well as remove the appropriate digital file form the OLD application's `files'
directory.  A confirm dialog appears in order to ensure that Files are not
mistakenly deleted.


\subsection{6.5.3   associate}
\label{user_guide:id9}
The associate button brings up the associate File page which allows one to
associate the File to one or more Forms.  Just enter the IDs (comma-separted) of
all Forms that should be associated to the File in question.


\subsection{6.5.4   export all}
\label{user_guide:id10}
When the File search results are displayed, there is a button near the top left
of the page labelled `export all'.  Clicking on this button will create an
archived folder (.zip) containing the digital files corresponding to the result
Files.  The page displayed will contain a link to the archived folder and a
listing of its contents.  Click on the link to download the archived folder.


\chapter{7   Collections}
\label{user_guide:collections}
Collections represent stories, records of elicitations, conversations or any
other multi-sentence discourse.  At its core, a Collection is a set of ordered
references to OLD Forms.  A Collection may also be associated to zero or more
Files (e.g., recordings of an entire story or elicitation session).


\section{7.1   Collection Data}
\label{user_guide:collection-data}

\subsection{7.1.1   ID}
\label{user_guide:id11}
The ID is a unique integer assigned by the RDBMS to each Collection upon
creation.  Knowing the ID of a Collection comes in handy when you want to
quickly access a Collection: enter `domain\_name/collection/view/1' to view the
Collection with ID 1.


\subsection{7.1.2   title}
\label{user_guide:title}
The title is a name for your Collection.  This field is mandatory.


\subsection{7.1.3   type}
\label{user_guide:type}
The type indicates what the Collection represents.  At present there are four
possible Collection types: story, discourse, elicitation and other.


\subsection{7.1.4   description}
\label{user_guide:id12}
The description contains general-purpose textual data describing the Collection.


\subsection{7.1.5   speaker}
\label{user_guide:id13}
If the Collection has a speaker, use this field to reference an OLD speaker for
the Collection.


\subsection{7.1.6   elicitor}
\label{user_guide:id14}
The elicitor field references the name of the researcher who elicited the
Collection, if appropriate.


\subsection{7.1.7   source}
\label{user_guide:id15}
If the Collection is from a textual source, one should use the source field to
make the reference.


\subsection{7.1.8   date elicited}
\label{user_guide:id16}
The date the Collection was elicited in mm/dd/yyyy format.


\subsection{7.1.9   content}
\label{user_guide:content}
The content of the Collection is a string of text.  That string can contain the
following types of markup:
\begin{itemize}
\item {} 
{\hyperref[user_guide:old-form-embed-markup]{OLD Form Embed Markup}}

\item {} 
{\hyperref[user_guide:old-entity-link-markup]{OLD Entity Link Markup}}

\item {} 
{\hyperref[user_guide:restructuredtext]{reStructuredText}}

\end{itemize}

Consider the following example Collection content:

\begin{Verbatim}[commandchars=\\\{\}]
Elicitation About Dogs
======================

The following sentence shows A, B and C.

form[1]

The above sentence was extracted from the audio recording in file(7)

The following sentence shows X and Y.

form[2]

Compare this discourse to that in collection(2)

The end.
\end{Verbatim}

The line underlined with `='s will be displayed as a level-one HTML header
and lines of text surrounded by blank lines will be displayed as HTML paragraphs
(see {\hyperref[user_guide:restructuredtext]{reStructuredText}}).  Depending on the data in Forms 1 and 2, this content
will be rendered similar to below:
\begin{figure}[htbp]
\centering
\capstart

\includegraphics{images/OLD_collection_content.png}
\caption{Illustration of how the content of a Collection is rendered}\end{figure}

Notice the following: `form{[}71{]}' and `form{[}72{]}' generate representations of the
appropriate forms; `\# Elicitation About Dogs' has been rendered as a header; and
`file(7)' and `collection(2)' have generated as links to the File 7 and
Collection 2 respectively.

When adding content to (or altering the content of) a Collection, there is a
button labelled `insert contents of memory'.  Clicking this button will append
to the content text a list of references to each Form in the user's memory.  It
is often convenient to gather relevant Forms into memory and then save them as
a Collection using this functionality.


\subsection{7.1.10   associated files}
\label{user_guide:associated-files}
Collections can be associated to zero or more Files.  To associate Files to a
Collection, use the `associate' button that appears when a Collection is
displayed.


\subsection{7.1.11   associated forms}
\label{user_guide:associated-forms}
Collections can be associated to zero or more Forms.  The Forms associated to a
particular Collection are those referenced in the content field of the
Collection.


\section{7.2   Viewing a Collection}
\label{user_guide:viewing-a-collection}
The content of a Collection can be displayed in three ways: long, short and
columns.  Long view displays the embedded Forms of a Collection's content just
as Forms are displayed by default, i.e., in interlinear gloss format with all of
their data visible.  Short view displays only the transcription and glosses of
embedded Forms.  Columns view is different from long and short in that the
embedded Forms are displayed but any prose or other markup is not displayed.
Columns view lists the Forms in a two-column table with the transcription on the
left and glosses on the right.


\section{7.3   Adding a Collection}
\label{user_guide:adding-a-collection}
To add a Collection, click `add' under `collections' in the side menu or use the
keyboard shortcut `z'.

The only obligatory field in a Collection is the title.  Enter the data as
necessary and format the Collection {\hyperref[user_guide:content]{content}} as described above.  Then click
``Add Collection''.


\section{7.4   Searching Collections}
\label{user_guide:searching-collections}
To search Collections, click `search' under `collections' in the side menu or
use the keyboard shortcut `x'.

Searching Collections is nearly identical to the process of searching Forms and
Files.  Refer to the {\hyperref[user_guide:searching-forms]{Searching Forms}} section for more details.


\section{7.5   Browsing Collections}
\label{user_guide:browsing-collections}
To browse all Collections, click `collections' in the side menu or use the
keyboard shortcut `c'.


\section{7.6   Other Collection Actions}
\label{user_guide:other-collection-actions}
When a Collection is displayed, three buttons appear: `update', `delete' and
`associate'.


\subsection{7.6.1   update}
\label{user_guide:id17}
The update button directs to the update Collection page which allows users to
alter the Collection's data.  Unlike with Forms, there is currently no
collection\_backup table to store previous versions of Collections.


\subsection{7.6.2   delete}
\label{user_guide:id18}
The delete button destroys the Collection.  Unlike when deleting Forms, deleted
Collections are not backed up in a collection\_backup table.


\subsection{7.6.3   associate}
\label{user_guide:id19}
The associate button allows users to associate one or more Files to the
Collection by entering a list of one or more comma-delimited File IDs.


\chapter{8   People}
\label{user_guide:people}
The `people' section lists information about the OLD entities representing
individuals, i.e., speakers and users.  To view the people of an OLD
application, click `database' in the primary menu and `people' in the secondary
menu (or use the keyboard shortcut `p').


\section{8.1   Speakers}
\label{user_guide:speakers}
Speakers are the consultants/colleagues who speak the object language and have
been kind enough to share their knowledge with the researchers who study and
document their language.

In order to document the speaker of a Form, File or Collection, that speaker
must first be represented as an OLD speaker entity.  The addition (or updating)
of a speaker immediately results in that speaker being available in the speaker
fields of Forms, Files and Collections.


\subsection{8.1.1   view a speaker}
\label{user_guide:view-a-speaker}
To view a speaker, click on the speaker's name in the speakers table of the
people page.


\subsection{8.1.2   add a speaker}
\label{user_guide:add-a-speaker}
Users (administrators and contributors) can add a new speaker by clicking the
`add' button.  A new speaker requires at least a first name and a last name.

If the object language has dialects, specifying the dialect spoken by the
speaker may be useful when other researchers are trying to interpret your data.

The page content section contains the text of the speaker's OLD web page.
{\hyperref[user_guide:restructuredtext]{reStructuredText}} syntax can be used to format the speaker's web page.  The page
content might include detailed information about the speaker's biography and
proficiency with the object language.


\subsection{8.1.3   update a speaker}
\label{user_guide:update-a-speaker}
To update the data about a speaker, click on the speaker's name in the speakers
table of the people page and click the `edit' button.


\subsection{8.1.4   delete a speaker}
\label{user_guide:delete-a-speaker}
To delete a speaker, click on the speaker's name in the speakers table of the
people page and click the `delete' button.  A confirmation dialog will arise to
ensure that speakers are not mistakenly deleted.  As a speaker may be referenced
by multiple OLD entities (some of them possibly entered by other users),
consider carefully the possible ramifications before deleting a speaker.


\section{8.2   Users}
\label{user_guide:users}
Users are the individuals who are authorized to access a particular OLD
application.  The list of users determines the individuals available as
enterers, elicitors and verifiers when adding Forms, Files and Collections.

There are three types of users with different levels of authorization:
administrators, contributors and viewers.  The permissions of these user types
are listed below


\subsection{8.2.1   administrators}
\label{user_guide:administrators}
Administrators are authorized to perform the following actions:
\begin{itemize}
\item {} 
view and update application settings

\item {} 
view, add, update and delete users

\item {} 
view, add, update and delete Forms, Files and Collections

\item {} 
view, add, update and delete speakers, tags and sources

\end{itemize}


\subsection{8.2.2   contributors}
\label{user_guide:contributors}
Contributors contribute data to an OLD application.  They are authorized to
perform the following actions:
\begin{itemize}
\item {} 
update their own user information

\item {} 
view other users' pages

\item {} 
view, add, update and delete Forms, Files and Collections

\item {} 
view, add, update and delete speakers, tags and sources

\end{itemize}


\subsection{8.2.3   viewers}
\label{user_guide:viewers}
Viewers can view data in an OLD application.  They are authorized to perform
the following actions:
\begin{itemize}
\item {} 
update their own user information

\item {} 
view other users' pages

\item {} 
view Forms, Files and Collections

\item {} 
view speakers, tags and sources

\end{itemize}


\subsection{8.2.4   view a user}
\label{user_guide:view-a-user}
To view a user, click on that user's name in the users table of the people page.


\subsection{8.2.5   add a user}
\label{user_guide:add-a-user}
Only administrators can add users.  To add a user, click the add button in the
users section of the people page.

New users require a username, a password, a first name, a last name, an email
and a role, i.e., administrator, contributor or viewer.

If appropriate, enter an affiliation, i.e., the name of a university, college,
school, community, organization, etc., with which the user is affiliated.

The personal page content is a string of text that comprises the OLD page of the
user.  The personal page content can contain the following types of markup:
\begin{itemize}
\item {} 
{\hyperref[user_guide:old-entity-link-markup]{OLD Entity Link Markup}}

\item {} 
{\hyperref[user_guide:old-file-file-markup]{OLD File file Markup}}

\item {} 
{\hyperref[user_guide:old-image-embed-markup]{OLD Image Embed Markup}}

\item {} 
{\hyperref[user_guide:restructuredtext]{reStructuredText}}

\end{itemize}


\subsection{8.2.6   edit a user}
\label{user_guide:edit-a-user}
Administrators can edit users and users can edit their own user data.  To edit
a user, click the user's name in the users table of the people page and then
click the edit button.


\subsection{8.2.7   delete a user}
\label{user_guide:delete-a-user}
Only administrators can delete users.  To delete a user, click the user's name
in the users table of the people page and then click the delete button.  A
confirm dialog will be displayed so that users are not mistakenly deleted.


\subsection{8.2.8   user settings}
\label{user_guide:user-settings}
Each user has their own settings page.  To view your settings, click on your
name in the people page and then click the settings button.  At present, one
can set a default view type for Collection contents (see) as well as a mode
(choice of database or dictionary).  (I don't think the mode setting actually
has any effect right now.  Make it do something or remove it!)


\chapter{9   Tags}
\label{user_guide:tags}
Tags are OLD entities which can be used to tag Forms and which are user-defined.
At present, there are three types of tag: keywords, syntactic categories and
elicitation methods.  Users of a particular OLD application can add new tags.
To view the tags, click on `database' in the primary menu and `tags' in the
secondary menu (or use the keyboard shortcut `t').


\section{9.1   Keywords}
\label{user_guide:id20}
Keywords are general-purpose tags that users can define.  A single form can be
associated to multiple keywords.  Some examples of potential keywords: `SVO',
`passive', `weak quantifier', etc.


\subsection{9.1.1   add a keyword}
\label{user_guide:add-a-keyword}
To add a new keyword, click the add button under the keywords heading of the
tags page.  A keyword requires a name.  The name is what will be listed in the
keywords field when adding/updating a Form.  Use the optional description field
to describe the keyword in detail.


\subsection{9.1.2   edit a keyword}
\label{user_guide:edit-a-keyword}
To edit a keyword, click the name of the keyword in the keywords table of the
tags page and then click edit.


\subsection{9.1.3   delete a keyword}
\label{user_guide:delete-a-keyword}
To delete a keyword, click the name of the keyword in the keywords table of the
tags page and then click delete.  A confirm dialog will appear to discourage
mistakenly deleting a keyword.  Use caution when deleting a keyword as doing
so will effect Forms associated to that keyword.


\section{9.2   Syntactic Categories}
\label{user_guide:syntactic-categories}
Syntactic categories are user-definable tags that are intended to be used to
indicate the morphological or syntactic category of the Form.  A Form can only
have one syntactic category.  Some examples of possible syntactic categories:
`S', `N', `V', `Asp', `Root', `Agr', etc.


\subsection{9.2.1   add a syntactic category}
\label{user_guide:add-a-syntactic-category}
To add a new syntactic category, click the add button under the syntactic
categories heading of the tags page.  A syntactic category requires a name.  The
name is what will be listed in the category field when adding/updating a Form.
Use the optional description field to describe the syntactic category in detail.


\subsection{9.2.2   edit a syntactic category}
\label{user_guide:edit-a-syntactic-category}
To edit a syntactic category, click the name of the syntactic category in the
syntactic categories table of the tags page and then click edit.


\subsection{9.2.3   delete a syntactic category}
\label{user_guide:delete-a-syntactic-category}
To delete a syntactic category, click the name of the syntactic category in the
syntactic categories table of the tags page and then click delete.  A confirm
dialog will appear to discourage mistakenly deleting a syntactic category.  Use
caution when deleting a syntactic category as doing so will effect Forms
associated to that syntactic category.


\section{9.3   Elicitation Methods}
\label{user_guide:elicitation-methods}
Elicitation methods are user-definable tags that are intended to be used to
indicate the manner in which the Form was elicited from a speaker.  A Form can
only have one elicitation method.  Some examples of possible elicitation
methods: `volunteered form', `used image stimulus', `part of a story', etc.


\subsection{9.3.1   add an elicitation method}
\label{user_guide:add-an-elicitation-method}
To add a new elicitation method, click the add button under the elicitation
methods heading of the tags page.  An elicitation method requires a name.  The
name is what will be listed in the elicitation method field when adding/updating
a Form.  Use the optional description field to describe the elicitation method
in detail.


\subsection{9.3.2   edit an elicitation method}
\label{user_guide:edit-an-elicitation-method}
To edit an elicitation method, click the name of the elicitation method in the
elicitation methods table of the tags page and then click edit.


\subsection{9.3.3   delete a syntactic category}
\label{user_guide:id21}
To delete an elicitation method, click the name of the elicitation method in the
elicitation methods table of the tags page and then click delete.  A confirm
dialog will appear to discourage mistakenly deleting an elicitation method.  Use
caution when deleting an elicitation method as doing so will effect Forms
associated to that elicitation method.


\chapter{10   Sources}
\label{user_guide:sources}
A source is an OLD entity representing a textual source, e.g., a book, a
dictionary, a paper, etc.  Both Forms and Collections can have sources.  To view
the sources of an OLD application, click `database' in the primary menu and then
`sources' in the secondary menu.


\section{10.1   Add a Source}
\label{user_guide:add-a-source}
To add a source, click the add button in the sources page.  Sources require the
following data: author first name, author last name, year of publication and
title.  Optional data include a full bibliographic reference for the source and
the ID of an OLD File representing a digital copy of the source.


\section{10.2   Edit a Source}
\label{user_guide:edit-a-source}
To edit a source, click the name of the source in the sources table of the
sources page and then click the edit button.


\section{10.3   Delete a Source}
\label{user_guide:delete-a-source}
To delete a source, click the name of the source in the sources table of the
sources page and then click the delete button.  A confirm dialog will appear in
order to avoid mistakenly deleting a source.  Use caution when deleting a source
since it may be referenced by Forms and Collections.


\chapter{11   Memory}
\label{user_guide:memory}
Each OLD application user has her own private memory.  Memory is simply an
ordered list of OLD Forms that a particular user has chosen to remember.  To
view your memory, click on `database' in the primary menu and then `memory' in
the secondary menu (or use the `m' keyboard shortcut).

At the top of the memory page are two buttons: `forget all' and `export all'.
Each Form in memory is displayed with its ID, transcription, morpheme break,
morpheme gloss and gloss values.  Beneath the Form data are three buttons:
`view', `export' and `forget'


\section{11.1   forget all}
\label{user_guide:forget-all}
The forget all button removes all Forms from memory.


\section{11.2   export all}
\label{user_guide:id22}
The export all button works just like the export all button that is displayed
with Form search results.  That is, clicking on this button will bring up the
Form export page from where a user may choose an export type, click `export' and
recieve their memorized Forms in the chosen format.


\chapter{12   Dictionary}
\label{user_guide:dictionary}
Clicking on `dictionary' in the primary menu (or using the `d' keyboard
shortcut) brings up a dictionary-like interface to the Forms in the database.
The dictionary excludes all Forms whose transcription contains a whitespace
character and in so doing makes a stab at getting only words.  The Forms listed
in the dictionary are sorted according to the order of graphs in the object
language orthography as specified in the OLD application's settings.


\section{12.1   Browse}
\label{user_guide:browse}
To browse the dictionary, click `browse' in the secondary menu.  The dictionary
browse page displays the object language name, a table of ordered graphs from
the object language orthography, the metalanguage name, and a table of ordered
graphs from the metalanguage orthography.


\subsection{12.1.1   object language to metalanguage}
\label{user_guide:object-language-to-metalanguage}
Clicking on the object language name will display all of the entries of the
dictionary in object-language-to-metalanguage format.  These entries will be
ordered according to the order of graphs in the object language orthography.

Clicking on an object language graph will display all dictionary entries that
begin with that graph.  These entries will be ordered according to the order of
graphs in the object language orthography.


\subsection{12.1.2   metalanguage to object language}
\label{user_guide:metalanguage-to-object-language}
Clicking on the metalanguage name will display all of the entries of the
dictionary in metalanguage-to-object-language format.  These entries will be
ordered according to the order of graphs in the metalanguage orthography.

Clicking on a metalanguage graph will display all dictionary entries that
begin with that graph.  These entries will be ordered according to the order of
graphs in the metalanguage orthography.


\section{12.2   Search}
\label{user_guide:search}
To search the dictionary, click `search' in the secondary menu.  The dictionary
search page displays a single text input field and a dropdown menu where one can
choose to search the object language or the metalanguage.


\subsection{12.2.1   object language to metalanguage}
\label{user_guide:id23}
If one enters a search term, chooses `object language to metalanguage' from the
dropdown menu and clicks `Search', the system will return all Forms with a
transcription which matches the search term and which lacks whitespace.  The
results displayed will be ordered according to the ordering of graphs in the
object language orthography.


\subsection{12.2.2   metalanguage to object language}
\label{user_guide:id24}
If one enters a search term, chooses `metalanguage to object language' from the
dropdown menu and clicks `Search', the system will return all Forms with a
gloss which matches the search term and with a transcription which lacks
whitespace.  The results displayed will be ordered according to the ordering of
graphs in the metalanguage orthography.


\chapter{13   Application Settings}
\label{user_guide:application-settings}
The application settings are the global settings for an OLD application. All
users can view the application settings but only administrators can alter
them.  To view the application settings, click `settings' in the primary menu.


\section{13.1   Object Language}
\label{user_guide:object-language}
The object language is the language that is being studied and documented with
the help of a particular OLD application.

Administrators should specify a name for the object language.  This name will be
used throughout the application to refer to the object language.

Administrators can also specify the ISO 639-3 three-letter code for the language
so that it can be unambiguously identified.  This is especially important if the
chosen object language name is one of many variants.  The OLD has all the ISO
639-3 codes stored.  When one begins to type in the ISO 639-3 Code text input
field, suggestions of valid codes appear along with the standard (`reference')
name of the corresponding language.  If the ISO 639-3 code cannot be found, try
searching the \href{http://www.ethnologue.com}{Ethnologue web site}.


\section{13.2   Storage, Input \& Output}
\label{user_guide:storage-input-output}
It is possible that an object language will have multiple orthographies in use.
In such a case, it would be nice to permit each user to interact with an OLD
application using the orthography of their choice.  On the other hand, it is
also desirable that all object language data be stored in the same orthography.
To resolve this conflict, the OLD facilitates automatic translation of object
language strings from one orthography to another.

For this reason, an OLD application requires the specification of three types
of object language orthography:
\begin{enumerate}
\item {} 
a storage orthography

\item {} 
a default input orthography

\item {} 
a default output orthography

\end{enumerate}

The storage orthography is immutable, but the input and output orthographies are
`default' because each user can override these settings in their user-specific
settings.  Henceforward, `input orthography' will refer to the user-specific
input orthography if specified, or the system-wide default otherwise.  Same
thing, mutatis mutandis, for `output orthography'.

The storage orthography is what input strings will be converted into for storage
and output strings converted from for display.  Object language input from the
user will be converted from the input orthography to the storage orthography.
Object language output will be converted from the storage orthography to the
output orthography for display.

The simplest setup (and the recommended one) is to specify a single orthography
for storage, input and output.  This will result in no conversions being made
when object language data are entered or displayed.  That way, whenever a user
wants to view or export (or enter) their data in a different orthography, they
can alter their user-specific settings as they please.

In special circumstances, it may be desirable to set the default input and
default output orthographies to orthography X and the storage orthography to
orthography Y.  (For example, X may be common and familiar to users yet Y is
chosen for storage because it is more expressive.)  With this setup, the OLD
application will handle all orthographic conversions and users using the
system defaults will not necessarily know (or need to know) about the existence
of the distinct storage orthography Y.


\section{13.3   Object Language Orthography(/ies)}
\label{user_guide:object-language-orthography-ies}
An object language orthography is an ordered list of graphs (characters or
character sequences) for representing object language data.  Depending on how
things are set up, an OLD application can be configured to automatically
transform object language data from one orthography to another (see
{\hyperref[user_guide:storage-input-output]{Storage, Input \& Output}} for details).

Administrators should specify at least one object language orthography.  As
discussed above ({\hyperref[user_guide:storage-input-output]{Storage, Input \& Output}}), the simplest setup is to specify
a single orthography and make that orthography be the storage, default input and
default output orthography.

If multiple object language orthographies are defined, then all orthographies
must have the same structure and the same number of graphs.  The OLD will not
allow multiple object language orthographies to be defined if these requirements
are not met.


\section{13.4   Object Language Data}
\label{user_guide:object-language-data}
Certain data types are assumed by the OLD to contain nothing but strings of
graphs from the input orthography; these I call `pure object language fields'.
Certain data types are optionally assumed to contain nothing but input
orthography strings; these I call `optionally object language fields'.  Finally,
some data types can contain object language tags (`\textless{}obl\textgreater{}\textless{}/obl\textgreater{}') and the data
withing such tags will be considered object language data; these fields I call
`object language tag fields'.


\subsection{13.4.1   Pure Object Language Fields}
\label{user_guide:pure-object-language-fields}
When entering data into, or when searching for data in, the pure object language
fields listed below, the input is converted from the input orthography to the
storage orthography.  When the data from such fields are displayed, they are
converted from the storage orthography to the output orthography.  (Also, when
data from these fields are displayed for updating, the data are converted from
the storage orthography to the input orthography.)

The following fields are pure object language fields:
\begin{itemize}
\item {} 
transcriptions of Forms

\end{itemize}


\subsection{13.4.2   Optionally Object Language Fields}
\label{user_guide:optionally-object-language-fields}
The following fields will be treated like pure object language fields if so
specified in the application settings.  To alter these settings, click on
`settings' in the primary menu, click the edit button and scroll down to the
`object language strings' fieldset.  In the dropdown menu, choose `yes' to treat
morpheme breaks as object language fields or `no' not to.
\begin{itemize}
\item {} 
morpheme breaks of Forms

\end{itemize}


\subsection{13.4.3   Object Language Tag Fields}
\label{user_guide:object-language-tag-fields}
The fields listed below may contain substrings enclosed in object language tags,
i.e., `\textless{}obl\textgreater{}' to the left and `\textless{}/obl\textgreater{}' to the right.  These substrings will be
treated like object language data for the purposes of orthographic conversion,
as described in the \emph{Pure Object Language Fields} section above.
\begin{itemize}
\item {} 
general comments of Forms

\item {} 
speaker comments of Forms

\end{itemize}


\section{13.5   Testing Orthography Conversion}
\label{user_guide:testing-orthography-conversion}
To test how well the OLD converts from one orthography to another, go to the
settings orthography page by clicking on `settings' in the primary menu, then
scroll down to the bottom of the `Storage, Input \& Output' section and click
the button labeled `More on Orthographies'.

This page allows users to do two things:
\begin{enumerate}
\item {} 
experiment with converting strings from one orthography to another

\item {} 
see whether they are able to enter the graphs of a particular orthography

\end{enumerate}

This page is pretty self-explanatory.


\section{13.6   Metalanguage}
\label{user_guide:metalanguage}
Administrators should specify the name, ISO 639-3 code and orthography of the
metalanguage as well.  The metalanguage name will be used throughout the system
to refer to the metalanguage and the metalanguage orthography will be used in
the dictionary interface to the database.


\chapter{14   Character Encoding}
\label{user_guide:character-encoding}
OLD applications handle all strings as unicode.  This means that users should
have no problems using obscure characters when entering data.


\chapter{15   Fonts}
\label{user_guide:fonts}
An OLD application may require the installation of certain fonts in order to
display characters properly.  If applicable, instructions for acquiring and
installing this font should be provided in the help page of this OLD
application (i.e., click on `help' in the primary menu and then `help page' in
the secondary menu).


\chapter{16   Regular Expressions}
\label{user_guide:regular-expressions}
Regular expressions allow one to specify complex patterns when searching Forms,
Files or Collections.  I will not attempt to provide a complete tutorial on how
to write regular expressions.  There are many online resources devoted to that
purpose.  Some places to start looking:
\begin{itemize}
\item {} 
\href{http://en.wikipedia.org/wiki/Regular\_expression}{wikipedia page on regular expressions}

\item {} 
\href{http://gnosis.cx/publish/programming/regular\_expressions.html}{gnosis.cx/publish/programming/regular\_expressions.html}

\end{itemize}


\section{16.1   Metacharacters}
\label{user_guide:metacharacters}
In the regular expression syntax, certain characters have special meanings.
These characters are called metacharacters.  In this section, I briefly review
the meanings of some of these characters.
\begin{description}
\item[{.}] \leavevmode
Matches any single character.  E.g., d.g matches `dog', `dig', `dkg', etc.
but not `doog'

\item[{{[}{]}}] \leavevmode
Matches any of the characters in the brackets.  E.g., d{[}iou{]}g matches `dog',
`dig' and `dug' and nothing else

\item[{{[}\textasciicircum{}{]}}] \leavevmode
Matches any character that is not in the brackets.  E.g., d{[}\textasciicircum{}ae{]}g matches
`dog', `dig', `dug', `dzg', etc. but not `dag', `deg' or `doog'

\end{description}
\begin{description}
\item[{\textasciicircum{}}] \leavevmode
Matches the beginning of the string.  E.g., \textasciicircum{}d matches `dog' and `dare him!'
but not ` dog' or `adrift'

\end{description}
\begin{description}
\item[{\$}] \leavevmode
Matches the end of the string.  E.g., k\$ matches `back' and `he will walk'
but not `back.', `he walks' or `kid'

\item[{*}] \leavevmode
Matches the preceding element zero or more times.  E.g., a*gh matches
`aaaaaaaaaagh', `he yelled ``aaagh''' and `light' but not `aaaagghh'.  (The
phrase `preceding element' is used to include subexpressions as well as
characters.  E.g., {[}ae{]}*gh matches `aagh!' or `eegh!')

\item[{+}] \leavevmode
Matches the preceding element one or more times.  E.g., a+gh matches
`aaaaaaaaaagh' and `he yelled ``aaagh''' but not `light'

\end{description}
\begin{description}
\item[{?}] \leavevmode
Matches the preceding element zero or one time.  E.g., bee?t matches `bet'
and `beet' and nothing else

\item[{\{m,n\}}] \leavevmode
Matches the preceding element at least m and no more than n times.  E.g.,
b{[}eo{]}\{1,2\}t matches `bet', `beet', `bot', `boot', `beot', `boet' and nothing
else

\item[{\{m\}}] \leavevmode
Matches the preceding element exactly m times

\item[{\{m,\}}] \leavevmode
Matches the preceding element at least m times

\item[{\{,m\}}] \leavevmode
Matches the preceding element no more than m times

\item[{\textbar{}}] \leavevmode
Matches the expression before the vertical bar or after it.  E.g., d(i\textbar{}o\textbar{}u)g
matches the exact same set of strings as d{[}iou{]}g does (see above).  E.g.,
using parentheses to group subexpressions, \textasciicircum{}((talk)\textbar{}(speak))s\$ matches
`talks' and `speaks' but nothing else

\end{description}


\section{16.2   Some Useful Regular Expressions}
\label{user_guide:some-useful-regular-expressions}\begin{description}
\item[{(\textasciicircum{}\textbar{} \textbar{}'\textbar{}'')word(\$\textbar{} \textbar{}'\textbar{}'')}] \leavevmode
Matches the word `word'.  I.e., matches `a bad word' and ```word'' was what
I said' without matching `words are interesting', `she is wordy' or
`sword', etc.

\item[{(\textasciicircum{}\textbar{} \textbar{}-\textbar{}=)mor(\$\textbar{} \textbar{}-\textbar{}=)}] \leavevmode
Matches the morpheme `mor'.  I.e., matches `a-mor=b' and `a=mor' without
matching `a-morp-b' or `kmor=b', etc.

\end{description}


\chapter{17   OLD Markup}
\label{user_guide:old-markup}
The OLD recognizes certain strings of characters as markup.  Such strings,
depending on where they occur, will be displayed in special ways.  For example,
the string `form{[}23{]}' in the content of a Collection will be displayed as a
formatted version of the OLD Form with ID 23.


\section{17.1   OLD Form Embed Markup}
\label{user_guide:old-form-embed-markup}
An expression of the form `form{[}x{]}', where `x' is the ID of a Form, will be
recognized as Form embed markup.  Form embed markup is recognized in
the content field of a Collection.  The expression `form{[}x{]}' in the content of
a Collection will be displayed as a formatted representation of the primary data
of the Form with ID x.

In addition, the expression `form{[}x{]}' will cause Form x to be listed as one of
the Forms that are associated to the Collection in question.


\section{17.2   OLD Entity Link Markup}
\label{user_guide:old-entity-link-markup}
An expression of the form `entity(x)', where `entity' is the name (in lowercase)
of an OLD entity (e.g., `form', `file', `collection', `speaker', `user') and `x'
is the ID of such an entity, will be recognized as Entity link markup.
Entity link markup creates HTML links to OLD entities.  That is,
expressions of the form `form(22)' `file(3102)' and `collection(77)' will be
displayed as HTML links to Form 22, File 3102 and Collection 77 respectively.
Entity link markup is recognized in the content of a Collection and in the
page content of users.


\section{17.3   OLD File file Markup}
\label{user_guide:old-file-file-markup}
An expression of the form `{[}x{]}link(y)', where `x' is a string of text and `y'
is the name of an OLD File, will be recognized as File file markup.  File file
markup creates an HTML link to the digital file of an OLD File.  That is,
the expression `{[}interesting audio{]}link(elicitation\_file.wav)' will be
displayed as a link consisting of the words `interesting audio' and which points
to the audio file elicitation\_file.wav.  File file markup is recognized in the
page content of users.


\section{17.4   OLD Image Embed Markup}
\label{user_guide:old-image-embed-markup}
An expression of the form `image(x)', where `x' is the name of an OLD image
File, will be recognized as image embed markup.  Image embed markup embeds the
appropriate image in the text.  That is, the expression
`image(stimulus\_photo.jpg)' will cause the image of the File whose name is
`stimulus\_photo.jpg' to be displayed.  Image embed markup is recognized in the
page content of users.


\section{17.5   OLD Metalanguage Markup}
\label{user_guide:old-metalanguage-markup}
The OLD expects certain fields (e.g., Form transcriptions and optionally
morpheme breaks) to contain data in the object language orthography.  Depending
on how the OLD application is set up, graphs of this orthography may be
displayed differently from how they are entered (see {\hyperref[user_guide:storage-input-output]{Storage, Input \& Output}}
above).  This can cause undesired character transformations if metalanguage
strings (e.g., names, borrowings) are entered in object language fields.  To
avoid this potential complication, the OLD permits the use of metalanguage
markup.  Expressions of the form `\textless{}ml\textgreater{}string\textless{}/ml\textgreater{}' will be interpreted as
metalanguage strings and the data between the `\textless{}ml\textgreater{}' and `\textless{}/ml\textgreater{}' tags will not
be treated as object language data, i.e., no orthographic conversions will be
applied.

For example, an OLD application might be set up to display the object language
string `ts' as `c'.  But if a Form transcription contains the English name
`Keats', it will be erroneously displayed as `Keac'.  To avoid this, simply
enter the metalanguage (in this case, English) string as `\textless{}ml\textgreater{}Keats\textless{}/ml\textgreater{}'.  This
will then be displayed as `Keats', as desired.


\chapter{18   reStructuredText}
\label{user_guide:restructuredtext}
reStructuredText is a human-readable markup syntax that can be converted to
HTML (as well as to LaTeX and other formats).  For example, the string:
\begin{quote}

\code{This is a Header}

\code{================}
\end{quote}

might be converted to `\textless{}h1\textgreater{}This is a Header\textless{}/h1\textgreater{}' and displayed by a browser as a
formatted top-level header. A good quick reference for reStructuredText syntax,
is the \href{http://docutils.sourceforge.net/docs/user/rst/quickref.html\#external-hyperlink-targets}{Quick reStructuredText web page}.

reStructuredText is the markup language of choice for creating OLD pages because
it can be converted to a variety of formats (i.e., HTML, LaTeX, ODT).


\chapter{19   Markdown}
\label{user_guide:markdown}
Markdown is a human-readable markup syntax that can be displayed as HTML.  For
example, the string `\# This is a Header' will be displayed as `\textless{}h1\textgreater{}This is a
Header\textless{}/h1\textgreater{}' which in turn will be displayed by the browser as a formatted
top-level header. For more details on Markdown syntax, visit the
\href{http://daringfireball.net/projects/markdown/syntax}{Markdown syntax web page}.


\chapter{20   Keyboard Shortcuts}
\label{user_guide:keyboard-shortcuts}
HTML allows the use of keyboard shortcuts on links (anchors) via the ``accesskey''
attribute.  The OLD uses accesskeys to create keyboard shortcuts for commonly
used actions.  Hold your cursor over an action link to see a description of what
the link does.  If that description is terminated by ``\textless{}x\textgreater{}'', then ``x'' is the
keyboard shortcut.

Accesskeys are implemented differently depending on the operating system and
browser being used.  With FireFox on a Mac, use Control + Key.  With Safari or
Chrome on a Mac, use Control + Option + Key.

List of keyboard shortcuts:
\begin{description}
\item[{f}] \leavevmode
browse all Forms

\item[{s}] \leavevmode
search Forms

\item[{a}] \leavevmode
add a Form

\item[{c}] \leavevmode
browse all Collections

\item[{x}] \leavevmode
search Collections

\item[{z}] \leavevmode
add a Collection

\item[{r}] \leavevmode
browse all Files

\item[{w}] \leavevmode
search Files

\item[{q}] \leavevmode
add a File

\item[{l}] \leavevmode
login or logout

\item[{p}] \leavevmode
view the people page

\item[{t}] \leavevmode
view the tags page

\item[{m}] \leavevmode
view the contents of one's memory

\end{description}


\chapter{21   Browser Support}
\label{user_guide:browser-support}
Web browsers do not render HTML, CSS and Javascript in a uniformly consistent
manner.  Therefore, parts of the OLD may not appear as intended, depending on
the OS and browser.

The OLD works on a Mac (OS 10.6) with FireFox, Safari and Chrome.  It has also
been tested on Ubuntu with FireFox without any apparent problems.  It has not
yet been tested with Internet Explorer on a PC.


\chapter{22   Glossary of Terms}
\label{user_guide:glossary-of-terms}\begin{description}
\item[{metalanguage}] \leavevmode
The language used to gloss and otherwise document the object language.

\item[{OLD}] \leavevmode
The Online Linguistic Database; a piece of software that can be used to
create OLD applications tailored to the documentation of particular
languages.

\item[{OLD application}] \leavevmode
A language-specific OLD instance.  For example, an OLD application used to
study Esperanto might be called the Esperanto Online Linguistic Database.

\item[{object language}] \leavevmode
The language under study.  The language that is being documented with the
help of a particular OLD application.

\item[{RDBMS}] \leavevmode
Relational Database Management System.  The database software, i.e., the
thing that allows us to store, retrieve and alter data.  An OLD application
may be configured to use the following RDBMSs: MySQL, SQLite or PostgreSQL.

\end{description}


\chapter{23   Future Features}
\label{user_guide:future-features}

\section{23.1   Priorities}
\label{user_guide:priorities}\begin{itemize}
\item {} 
administrators/users should be able to alter/add to the list of possible
grammaticalities/glossGrammaticalities and possible morpheme delimiters

\item {} 
offer further export formats (sql, odt) (docutils offers odt creation!)

\item {} 
clean up code (docstrings in controllers, make more readable, etc.)

\item {} 
finish documentation (developer guide, api, user guide)

\item {} 
is \textless{}ml\textgreater{}\textless{}/ml\textgreater{} markup available in the morpheme break field?  If not, it should
be

\item {} 
test the OLD on a PC with various browsers and update the OLD User Guide

\end{itemize}


\section{23.2   Relatively Easy Projects}
\label{user_guide:relatively-easy-projects}\begin{itemize}
\item {} 
when entering data, certain values from past additions should be made default,
i.e., elicitor and syntactic category are often repeatedly the same person
when a user is entering multiple forms

\item {} 
currently all searches are case insensitive (should this be changed?...)

\item {} 
ability to search Forms by association to Files and vice versa

\item {} 
allow the file of a File to be hosted on another server, e.g., use the
embeddedFileMarkup column of the file table

\item {} 
make forms searchable by dialect (or is this redundant since one can already
search by (sets of) speaker(s)?)

\item {} 
create `forgot my password' functionality (and forgot my username
functionality too?)

\item {} 
I think an administrator might be able to delete her own user account.  This
should be prohibited

\end{itemize}


\section{23.3   Big Projects}
\label{user_guide:big-projects}\begin{itemize}
\item {} 
integrate a general-purpose morphological parser (SFST, pySFST)

\item {} 
ability to indicate vowel and consonant classes on orthographies

\item {} 
warnings or errors when orthographically invalid transcriptions/morpheme
breaks are entered

\item {} 
orthography-specific ordering should be available via the database interface

\item {} 
forums/discussion functionality

\item {} 
create functionality that will allow users to create an arbitrary number of
web pages (worth it?)

\end{itemize}


\chapter{Developer Guide}
\label{developer_guide::doc}\label{developer_guide:ethnologue-web-site}\label{developer_guide:developer-guide}
This software is documented in detail in the SimpleSite tutorial chapters of the
book \emph{The Definitive Guide to Pylons} available under an open source license at
\href{http://pylonsbook.com}{http://pylonsbook.com}. You should read those chapters to discover how SimpleSite
is developed.


\chapter{API Documentation}
\label{api:api-documentation}\label{api::doc}
This page contains some basic documentation for the SimpleSite project. To
understand the project completely please refer to the documentation on the
Pylons Book website at \href{http://pylonsbook.com}{http://pylonsbook.com} or read the source code directly.


\section{The \texttt{simplesite} Module}
\label{api:module-simplesite}\label{api:the-simplesite-module}\index{simplesite (module)}
Contains all the controllers, model and templates as sub-modules.


\section{The \texttt{controllers} Module}
\label{api:the-controllers-module}\label{api:module-simplesite.controllers}\index{simplesite.controllers (module)}
Contains all the controllers. The most important of which is
{\hyperref[api:simplesite.controllers.PageController]{\code{PageController}}}.
\index{PageController (class in simplesite.controllers)}

\begin{fulllineitems}
\phantomsection\label{api:simplesite.controllers.PageController}\pysigline{\strong{class }\code{simplesite.controllers.}\bfcode{PageController}}
\end{fulllineitems}


The {\hyperref[api:simplesite.controllers.PageController]{\code{PageController}}} is responsible for displaying pages as well as
allowing users to add, edit, delete and list pages.
\index{view() (simplesite.controllers.PageController method)}

\begin{fulllineitems}
\phantomsection\label{api:simplesite.controllers.PageController.view}\pysiglinewithargsret{\code{PageController.}\bfcode{view}}{\emph{self}\optional{, \emph{id=None}}}{}
\end{fulllineitems}


When a user visits a URL such as \code{/view/page/1} the {\hyperref[api:simplesite.controllers.PageController]{\code{PageController}}}
class's \code{view()} action is called to render the page.

The page controller makes use of a FormEncode schema to validate the page data
it receives. Here is the schema it uses:

\begin{Verbatim}[commandchars=\\\{\}]
\PYG{k}{class} \PYG{n+nc}{NewPageForm}\PYG{p}{(}\PYG{n}{formencode}\PYG{o}{.}\PYG{n}{Schema}\PYG{p}{)}\PYG{p}{:}
    \PYG{n}{allow\PYGZus{}extra\PYGZus{}fields} \PYG{o}{=} \PYG{n+nb+bp}{True}
    \PYG{n}{filter\PYGZus{}extra\PYGZus{}fields} \PYG{o}{=} \PYG{n+nb+bp}{True}
    \PYG{n}{content} \PYG{o}{=} \PYG{n}{formencode}\PYG{o}{.}\PYG{n}{validators}\PYG{o}{.}\PYG{n}{String}\PYG{p}{(}
        \PYG{n}{not\PYGZus{}empty}\PYG{o}{=}\PYG{n+nb+bp}{True}\PYG{p}{,}
        \PYG{n}{messages}\PYG{o}{=}\PYG{p}{\PYGZob{}}
            \PYG{l+s}{'}\PYG{l+s}{empty}\PYG{l+s}{'}\PYG{p}{:}\PYG{l+s}{'}\PYG{l+s}{Please enter some content for the page. }\PYG{l+s}{'}
        \PYG{p}{\PYGZcb{}}
    \PYG{p}{)}
    \PYG{n}{heading} \PYG{o}{=} \PYG{n}{formencode}\PYG{o}{.}\PYG{n}{validators}\PYG{o}{.}\PYG{n}{String}\PYG{p}{(}\PYG{p}{)}
    \PYG{n}{title} \PYG{o}{=} \PYG{n}{formencode}\PYG{o}{.}\PYG{n}{validators}\PYG{o}{.}\PYG{n}{String}\PYG{p}{(}\PYG{n}{not\PYGZus{}empty}\PYG{o}{=}\PYG{n+nb+bp}{True}\PYG{p}{)}
\end{Verbatim}

As you can see the schema includes validators for the title, heading and content.


\section{The \texttt{utils} Module}
\label{api:the-utils-module}\label{api:module-old.lib.utils}\index{old.lib.utils (module)}

\subsection{The utils module is really cool!}
\label{api:the-utils-module-is-really-cool}
\begin{Verbatim}[commandchars=\\\{\}]
\PYG{k}{class} \PYG{n+nc}{NewPageForm}\PYG{p}{(}\PYG{n}{formencode}\PYG{o}{.}\PYG{n}{Schema}\PYG{p}{)}\PYG{p}{:}
    \PYG{n}{allow\PYGZus{}extra\PYGZus{}fields} \PYG{o}{=} \PYG{n+nb+bp}{True}
    \PYG{n}{filter\PYGZus{}extra\PYGZus{}fields} \PYG{o}{=} \PYG{n+nb+bp}{True}
    \PYG{n}{content} \PYG{o}{=} \PYG{n}{formencode}\PYG{o}{.}\PYG{n}{validators}\PYG{o}{.}\PYG{n}{String}\PYG{p}{(}
        \PYG{n}{not\PYGZus{}empty}\PYG{o}{=}\PYG{n+nb+bp}{True}\PYG{p}{,}
        \PYG{n}{messages}\PYG{o}{=}\PYG{p}{\PYGZob{}}
            \PYG{l+s}{'}\PYG{l+s}{empty}\PYG{l+s}{'}\PYG{p}{:}\PYG{l+s}{'}\PYG{l+s}{Please enter some content for the page. }\PYG{l+s}{'}
        \PYG{p}{\PYGZcb{}}
    \PYG{p}{)}
    \PYG{n}{heading} \PYG{o}{=} \PYG{n}{formencode}\PYG{o}{.}\PYG{n}{validators}\PYG{o}{.}\PYG{n}{String}\PYG{p}{(}\PYG{p}{)}
    \PYG{n}{title} \PYG{o}{=} \PYG{n}{formencode}\PYG{o}{.}\PYG{n}{validators}\PYG{o}{.}\PYG{n}{String}\PYG{p}{(}\PYG{n}{not\PYGZus{}empty}\PYG{o}{=}\PYG{n+nb+bp}{True}\PYG{p}{)}
\end{Verbatim}
\index{ApplicationSettings (class in old.lib.utils)}

\begin{fulllineitems}
\phantomsection\label{api:old.lib.utils.ApplicationSettings}\pysigline{\strong{class }\code{old.lib.utils.}\bfcode{ApplicationSettings}}
ApplicationSettings is a class that adds functionality to a
ApplicationSettings object.

The value of the applicationSettings attribute is the most recently added
ApplicationSettings model.  Other values, e.g., storageOrthography or
morphemeBreakInventory, are class instances or other data structures built
upon the application settings properties.
\index{getAttributes() (old.lib.utils.ApplicationSettings method)}

\begin{fulllineitems}
\phantomsection\label{api:old.lib.utils.ApplicationSettings.getAttributes}\pysiglinewithargsret{\bfcode{getAttributes}}{}{}
Generate some higher-level data structures for the application
settings model, providing sensible defaults where appropriate.

\end{fulllineitems}


\end{fulllineitems}

\index{EventHook (class in old.lib.utils)}

\begin{fulllineitems}
\phantomsection\label{api:old.lib.utils.EventHook}\pysigline{\strong{class }\code{old.lib.utils.}\bfcode{EventHook}}
EventHook is for event-based (PubSub) stuff in Python.  It is taken from
\href{http://www.voidspace.org.uk/python/weblog/arch\_d7\_2007\_02\_03.shtml\#e616}{http://www.voidspace.org.uk/python/weblog/arch\_d7\_2007\_02\_03.shtml\#e616}.
See also \href{http://stackoverflow.com/questions/1092531/event-system-in-python}{http://stackoverflow.com/questions/1092531/event-system-in-python}.

\end{fulllineitems}

\index{Inventory (class in old.lib.utils)}

\begin{fulllineitems}
\phantomsection\label{api:old.lib.utils.Inventory}\pysiglinewithargsret{\strong{class }\code{old.lib.utils.}\bfcode{Inventory}}{\emph{inputList}}{}
An inventory is a set of graphemes/polygraphs/characters.  Initialization
requires a list.

This class should be the base class from which the Orthography class
inherits but I don't have time to implement that right now.
\index{getNonMatchingSubstrings() (old.lib.utils.Inventory method)}

\begin{fulllineitems}
\phantomsection\label{api:old.lib.utils.Inventory.getNonMatchingSubstrings}\pysiglinewithargsret{\bfcode{getNonMatchingSubstrings}}{\emph{string}}{}
Return a list of substrings of string that are not constructable
using the inventory.  This is useful for showing invalid substrings.

\end{fulllineitems}

\index{getRegexValidator() (old.lib.utils.Inventory method)}

\begin{fulllineitems}
\phantomsection\label{api:old.lib.utils.Inventory.getRegexValidator}\pysiglinewithargsret{\bfcode{getRegexValidator}}{\emph{substr=False}}{}
Returns a regex that matches only strings composed of zero or more
of the graphemes in the inventory (plus the space character).

\end{fulllineitems}

\index{stringIsValid() (old.lib.utils.Inventory method)}

\begin{fulllineitems}
\phantomsection\label{api:old.lib.utils.Inventory.stringIsValid}\pysiglinewithargsret{\bfcode{stringIsValid}}{\emph{string}}{}
Return False if string cannot be generated by concatenating the
elements of the orthography; otherwise, return True.

\end{fulllineitems}


\end{fulllineitems}

\index{JSONOLDEncoder (class in old.lib.utils)}

\begin{fulllineitems}
\phantomsection\label{api:old.lib.utils.JSONOLDEncoder}\pysiglinewithargsret{\strong{class }\code{old.lib.utils.}\bfcode{JSONOLDEncoder}}{\emph{skipkeys=False}, \emph{ensure\_ascii=True}, \emph{check\_circular=True}, \emph{allow\_nan=True}, \emph{sort\_keys=False}, \emph{indent=None}, \emph{separators=None}, \emph{encoding='utf-8'}, \emph{default=None}, \emph{use\_decimal=True}, \emph{namedtuple\_as\_object=True}, \emph{tuple\_as\_array=True}, \emph{bigint\_as\_string=False}, \emph{item\_sort\_key=None}}{}
Permits the jsonification of an OLD class instance obj via
\begin{quote}

jsonString = json.dumps(obj, cls=JSONOLDEncoder)
\end{quote}

Note: support for additional OLD classes will be implemented as needed ...

\end{fulllineitems}

\index{OrderBySchema (class in old.lib.utils)}

\begin{fulllineitems}
\phantomsection\label{api:old.lib.utils.OrderBySchema}\pysiglinewithargsret{\strong{class }\code{old.lib.utils.}\bfcode{OrderBySchema}}{\emph{*args}, \emph{**kw}}{}
\textbf{Messages}
\begin{description}
\item[{\code{badDictType}:}] \leavevmode
The input must be dict-like (not a \code{\%(type)s}: \code{\%(value)r})

\item[{\code{badType}:}] \leavevmode
The input must be a string (not a \code{\%(type)s}: \code{\%(value)r})

\item[{\code{empty}:}] \leavevmode
Please enter a value

\item[{\code{missingValue}:}] \leavevmode
Missing value

\item[{\code{noneType}:}] \leavevmode
The input must be a string (not None)

\item[{\code{notExpected}:}] \leavevmode
The input field \code{\%(name)s} was not expected.

\end{description}

\end{fulllineitems}

\index{PaginatorSchema (class in old.lib.utils)}

\begin{fulllineitems}
\phantomsection\label{api:old.lib.utils.PaginatorSchema}\pysiglinewithargsret{\strong{class }\code{old.lib.utils.}\bfcode{PaginatorSchema}}{\emph{*args}, \emph{**kw}}{}
\textbf{Messages}
\begin{description}
\item[{\code{badDictType}:}] \leavevmode
The input must be dict-like (not a \code{\%(type)s}: \code{\%(value)r})

\item[{\code{badType}:}] \leavevmode
The input must be a string (not a \code{\%(type)s}: \code{\%(value)r})

\item[{\code{empty}:}] \leavevmode
Please enter a value

\item[{\code{missingValue}:}] \leavevmode
Missing value

\item[{\code{noneType}:}] \leavevmode
The input must be a string (not None)

\item[{\code{notExpected}:}] \leavevmode
The input field \code{\%(name)s} was not expected.

\end{description}

\end{fulllineitems}

\index{State (class in old.lib.utils)}

\begin{fulllineitems}
\phantomsection\label{api:old.lib.utils.State}\pysigline{\strong{class }\code{old.lib.utils.}\bfcode{State}}
Empty class used to create a state instance with a `full\_dict' attribute
that points to a dict of values being validated by a schema.  For example,
the call to FormSchema().to\_python in controllers/forms.py requires this
State() instance as its second argument in order to make the inventory-based
validators work correctly (see, e.g., ValidOrthographicTranscription).

\end{fulllineitems}

\index{addOrderBy() (in module old.lib.utils)}

\begin{fulllineitems}
\phantomsection\label{api:old.lib.utils.addOrderBy}\pysiglinewithargsret{\code{old.lib.utils.}\bfcode{addOrderBy}}{\emph{query}, \emph{orderByParams}, \emph{queryBuilder}, \emph{primaryKey='id'}}{}
Add an ORDER BY clause to the query using the getSQLAOrderBy method of
the supplied queryBuilder (if possible) or using a default ORDER BY \textless{}primaryKey\textgreater{} ASC.

\end{fulllineitems}

\index{clearAllModels() (in module old.lib.utils)}

\begin{fulllineitems}
\phantomsection\label{api:old.lib.utils.clearAllModels}\pysiglinewithargsret{\code{old.lib.utils.}\bfcode{clearAllModels}}{\emph{retain={[}'Language'{]}}}{}
Convenience function for removing all OLD models from the database.
The retain parameter is a list of model names that should not be cleared.

\end{fulllineitems}

\index{clearDirectoryOfFiles() (in module old.lib.utils)}

\begin{fulllineitems}
\phantomsection\label{api:old.lib.utils.clearDirectoryOfFiles}\pysiglinewithargsret{\code{old.lib.utils.}\bfcode{clearDirectoryOfFiles}}{\emph{directoryPath}}{}
Removes all files from the directory path but leaves the directory.

\end{fulllineitems}

\index{commandLineProgramInstalled() (in module old.lib.utils)}

\begin{fulllineitems}
\phantomsection\label{api:old.lib.utils.commandLineProgramInstalled}\pysiglinewithargsret{\code{old.lib.utils.}\bfcode{commandLineProgramInstalled}}{\emph{command}}{}
Command is the list representing the command-line utility.

\end{fulllineitems}

\index{compile\_query() (in module old.lib.utils)}

\begin{fulllineitems}
\phantomsection\label{api:old.lib.utils.compile_query}\pysiglinewithargsret{\code{old.lib.utils.}\bfcode{compile\_query}}{\emph{query}, \emph{**kwargs}}{}
Return the SQLAlchemy query as a bona fide MySQL query.  Taken from
\href{http://stackoverflow.com/questions/4617291/how-do-i-get-a-raw-compiled-sql-query-from-a-sqlalchemy-expression}{http://stackoverflow.com/questions/4617291/how-do-i-get-a-raw-compiled-sql-query-from-a-sqlalchemy-expression}.

\end{fulllineitems}

\index{createResearcherDirectory() (in module old.lib.utils)}

\begin{fulllineitems}
\phantomsection\label{api:old.lib.utils.createResearcherDirectory}\pysiglinewithargsret{\code{old.lib.utils.}\bfcode{createResearcherDirectory}}{\emph{researcher}, \emph{**kwargs}}{}
Creates a directory named researcher.username in files/researchers/.

\end{fulllineitems}

\index{dateString2date() (in module old.lib.utils)}

\begin{fulllineitems}
\phantomsection\label{api:old.lib.utils.dateString2date}\pysiglinewithargsret{\code{old.lib.utils.}\bfcode{dateString2date}}{\emph{dateString}}{}
Parse an ISO 8601-formatted date into a Python date object.

\end{fulllineitems}

\index{datetimeString2datetime() (in module old.lib.utils)}

\begin{fulllineitems}
\phantomsection\label{api:old.lib.utils.datetimeString2datetime}\pysiglinewithargsret{\code{old.lib.utils.}\bfcode{datetimeString2datetime}}{\emph{datetimeString}}{}
Parse an ISO 8601-formatted datetime into a Python datetime object.
Cf. \href{http://stackoverflow.com/questions/531157/parsing-datetime-strings-with-microseconds}{http://stackoverflow.com/questions/531157/parsing-datetime-strings-with-microseconds}

Previously called ISO8601Str2datetime.

\end{fulllineitems}

\index{deleteKey() (in module old.lib.utils)}

\begin{fulllineitems}
\phantomsection\label{api:old.lib.utils.deleteKey}\pysiglinewithargsret{\code{old.lib.utils.}\bfcode{deleteKey}}{\emph{dict\_}, \emph{key\_}}{}
Try to delete the {\color{red}\bfseries{}key\_} from the {\color{red}\bfseries{}dict\_}; then return the {\color{red}\bfseries{}dict\_}.

\end{fulllineitems}

\index{destroyAllResearcherDirectories() (in module old.lib.utils)}

\begin{fulllineitems}
\phantomsection\label{api:old.lib.utils.destroyAllResearcherDirectories}\pysiglinewithargsret{\code{old.lib.utils.}\bfcode{destroyAllResearcherDirectories}}{\emph{**kwargs}}{}
Removes all directories from files/researchers/.

\end{fulllineitems}

\index{destroyResearcherDirectory() (in module old.lib.utils)}

\begin{fulllineitems}
\phantomsection\label{api:old.lib.utils.destroyResearcherDirectory}\pysiglinewithargsret{\code{old.lib.utils.}\bfcode{destroyResearcherDirectory}}{\emph{researcher}, \emph{**kwargs}}{}
Destroys a directory named researcher.username in files/researchers/.

\end{fulllineitems}

\index{encryptPassword() (in module old.lib.utils)}

\begin{fulllineitems}
\phantomsection\label{api:old.lib.utils.encryptPassword}\pysiglinewithargsret{\code{old.lib.utils.}\bfcode{encryptPassword}}{\emph{password}, \emph{salt}}{}
Use PassLib's pbkdf2 implementation to generate a hash from a password.
Cf. \href{http://packages.python.org/passlib/lib/passlib.hash.pbkdf2\_digest.html\#passlib.hash.pbkdf2\_sha512}{http://packages.python.org/passlib/lib/passlib.hash.pbkdf2\_digest.html\#passlib.hash.pbkdf2\_sha512}

\end{fulllineitems}

\index{escREMetaChars() (in module old.lib.utils)}

\begin{fulllineitems}
\phantomsection\label{api:old.lib.utils.escREMetaChars}\pysiglinewithargsret{\code{old.lib.utils.}\bfcode{escREMetaChars}}{\emph{string}}{}
Escapes regex metacharacters so that we can formulate an SQL regular
expression based on an arbitrary, user-specified inventory of
graphemes/polygraphs.

\begin{Verbatim}[commandchars=\\\{\}]
\PYG{g+gp}{\PYGZgt{}\PYGZgt{}\PYGZgt{} }\PYG{n}{escREMetaChars}\PYG{p}{(}\PYG{l+s}{u'}\PYG{l+s}{-}\PYG{l+s}{'}\PYG{p}{)}
\PYG{g+go}{u'\PYGZbs{}\PYGZbs{}-'}
\end{Verbatim}

\end{fulllineitems}

\index{ffmpegEncodes() (in module old.lib.utils)}

\begin{fulllineitems}
\phantomsection\label{api:old.lib.utils.ffmpegEncodes}\pysiglinewithargsret{\code{old.lib.utils.}\bfcode{ffmpegEncodes}}{\emph{format\_}}{}
Check if ffmpeg encodes the input format.  First check if it's installed.

\end{fulllineitems}

\index{ffmpegInstalled() (in module old.lib.utils)}

\begin{fulllineitems}
\phantomsection\label{api:old.lib.utils.ffmpegInstalled}\pysiglinewithargsret{\code{old.lib.utils.}\bfcode{ffmpegInstalled}}{}{}
Check if the ffmpeg command-line utility is installed on the host.  Check
first if the answer to this question is cached in app\_globals.

\end{fulllineitems}

\index{getCollectionBackupsByCollectionId() (in module old.lib.utils)}

\begin{fulllineitems}
\phantomsection\label{api:old.lib.utils.getCollectionBackupsByCollectionId}\pysiglinewithargsret{\code{old.lib.utils.}\bfcode{getCollectionBackupsByCollectionId}}{\emph{collectionId}}{}
Return all CollectionBackup models with collection\_id = collectionId.
WARNING: unexpected data may be returned (on an SQLite backend) if primary
key ids of deleted collections are recycled.

\end{fulllineitems}

\index{getCollectionBackupsByUUID() (in module old.lib.utils)}

\begin{fulllineitems}
\phantomsection\label{api:old.lib.utils.getCollectionBackupsByUUID}\pysiglinewithargsret{\code{old.lib.utils.}\bfcode{getCollectionBackupsByUUID}}{\emph{UUID}}{}
Return all CollectionBackup models with UUID = UUID.

\end{fulllineitems}

\index{getCollectionByUUID() (in module old.lib.utils)}

\begin{fulllineitems}
\phantomsection\label{api:old.lib.utils.getCollectionByUUID}\pysiglinewithargsret{\code{old.lib.utils.}\bfcode{getCollectionByUUID}}{\emph{UUID}}{}
Return the first (and only, hopefully) Collection model with UUID.

\end{fulllineitems}

\index{getConfig() (in module old.lib.utils)}

\begin{fulllineitems}
\phantomsection\label{api:old.lib.utils.getConfig}\pysiglinewithargsret{\code{old.lib.utils.}\bfcode{getConfig}}{\emph{**kwargs}}{}
Try desperately to get a Pylons config object.  The best thing is if a
config object is passed in kwargs{[}'config'{]}.

\end{fulllineitems}

\index{getDataForNewAction() (in module old.lib.utils)}

\begin{fulllineitems}
\phantomsection\label{api:old.lib.utils.getDataForNewAction}\pysiglinewithargsret{\code{old.lib.utils.}\bfcode{getDataForNewAction}}{\emph{GET\_params}, \emph{getterMap}, \emph{modelNameMap}}{}
Return a dictionary whose values are lists of OLD model objects.
GET\_params is the dict-like object created by Pylons that is created on a
GET request.  The getterMap param is a dict from key names (e.g., `users')
to a getter function that retrieves that resource (e.g., getUsers).  The
modelNameMap is a dict from key names (e.g., `users') to the relevant model
(e.g., `User').

If no GET parameters are provided (i.e., GET\_params is empty), then retrieve
all data (using getterMap) from the db and return them.

If GET parameters are specified, then for each parameter whose value is a
non-empty string (and is not a valid ISO 8601 datetime), retrieve and
return the appropriate list of objects.

If the value of a GET parameter is a valid ISO 8601 datetime string,
retrieve and return the appropriate list of objects \emph{only} if the
datetime param does \emph{not} match the most recent datetimeModified value
of the relevant data store (i.e., model object).  This makes sense because a
non-match indicates that the requester has out-of-date data.

\end{fulllineitems}

\index{getForeignWordTranscriptions() (in module old.lib.utils)}

\begin{fulllineitems}
\phantomsection\label{api:old.lib.utils.getForeignWordTranscriptions}\pysiglinewithargsret{\code{old.lib.utils.}\bfcode{getForeignWordTranscriptions}}{}{}
Returns a 4-tuple (fWNarrPhonTranscrs, fWBroadPhonTranscrs,
fWOrthTranscrs, fWMorphTranscrs) where each element is a list of
transcriptions (narrow phonetic, broad phonetic, orthographic, morphemic)
of foreign words.

\end{fulllineitems}

\index{getForeignWords() (in module old.lib.utils)}

\begin{fulllineitems}
\phantomsection\label{api:old.lib.utils.getForeignWords}\pysiglinewithargsret{\code{old.lib.utils.}\bfcode{getForeignWords}}{}{}
Return the forms that are tagged with a `foreign word' tag.  This is
useful for input validation as foreign words may contain otherwise illicit
characters/graphemes.

\end{fulllineitems}

\index{getFormBackupsByFormId() (in module old.lib.utils)}

\begin{fulllineitems}
\phantomsection\label{api:old.lib.utils.getFormBackupsByFormId}\pysiglinewithargsret{\code{old.lib.utils.}\bfcode{getFormBackupsByFormId}}{\emph{formId}}{}
Return all FormBackup models with form\_id = formId.  WARNING: unexpected
data may be returned (on an SQLite backend) if primary key ids of deleted
forms are recycled.

\end{fulllineitems}

\index{getFormBackupsByUUID() (in module old.lib.utils)}

\begin{fulllineitems}
\phantomsection\label{api:old.lib.utils.getFormBackupsByUUID}\pysiglinewithargsret{\code{old.lib.utils.}\bfcode{getFormBackupsByUUID}}{\emph{UUID}}{}
Return all FormBackup models with UUID = UUID.

\end{fulllineitems}

\index{getFormByUUID() (in module old.lib.utils)}

\begin{fulllineitems}
\phantomsection\label{api:old.lib.utils.getFormByUUID}\pysiglinewithargsret{\code{old.lib.utils.}\bfcode{getFormByUUID}}{\emph{UUID}}{}
Return the first (and only, hopefully) Form model with UUID.

\end{fulllineitems}

\index{getMorphemeDelimiters() (in module old.lib.utils)}

\begin{fulllineitems}
\phantomsection\label{api:old.lib.utils.getMorphemeDelimiters}\pysiglinewithargsret{\code{old.lib.utils.}\bfcode{getMorphemeDelimiters}}{}{}
Return the morpheme delimiters from app settings as a list.

\end{fulllineitems}

\index{getMostRecentModificationDatetime() (in module old.lib.utils)}

\begin{fulllineitems}
\phantomsection\label{api:old.lib.utils.getMostRecentModificationDatetime}\pysiglinewithargsret{\code{old.lib.utils.}\bfcode{getMostRecentModificationDatetime}}{\emph{modelName}}{}
Return the most recent datetimeModified attribute for the model with the
provided modelName.  If the modelName is not recognized, return None.

\end{fulllineitems}

\index{getSearchParameters() (in module old.lib.utils)}

\begin{fulllineitems}
\phantomsection\label{api:old.lib.utils.getSearchParameters}\pysiglinewithargsret{\code{old.lib.utils.}\bfcode{getSearchParameters}}{\emph{queryBuilder}}{}
Given an SQLAQueryBuilder instance, return (relative to the model being
searched) the list of attributes and their aliases and licit relations
relevant to searching.

\end{fulllineitems}

\index{getStateObject() (in module old.lib.utils)}

\begin{fulllineitems}
\phantomsection\label{api:old.lib.utils.getStateObject}\pysiglinewithargsret{\code{old.lib.utils.}\bfcode{getStateObject}}{\emph{values}}{}
Return a State instance with some special attributes needed in the forms
and oldcollections controllers.

\end{fulllineitems}

\index{getSubprocess() (in module old.lib.utils)}

\begin{fulllineitems}
\phantomsection\label{api:old.lib.utils.getSubprocess}\pysiglinewithargsret{\code{old.lib.utils.}\bfcode{getSubprocess}}{\emph{command}}{}
Return a subprocess process.  The command argument is a list.  See
\href{http://docs.python.org/2/library/subprocess.html}{http://docs.python.org/2/library/subprocess.html}

\end{fulllineitems}

\index{getUnicodeCodePoints() (in module old.lib.utils)}

\begin{fulllineitems}
\phantomsection\label{api:old.lib.utils.getUnicodeCodePoints}\pysiglinewithargsret{\code{old.lib.utils.}\bfcode{getUnicodeCodePoints}}{\emph{string}}{}
Returns a string of comma-delimited unicode code points corresponding
to the characters in the input string.

\end{fulllineitems}

\index{getUnicodeNames() (in module old.lib.utils)}

\begin{fulllineitems}
\phantomsection\label{api:old.lib.utils.getUnicodeNames}\pysiglinewithargsret{\code{old.lib.utils.}\bfcode{getUnicodeNames}}{\emph{string}}{}
Returns a string of comma-delimited unicode character names corresponding
to the characters in the input string.

\end{fulllineitems}

\index{getUnrestrictedUsers() (in module old.lib.utils)}

\begin{fulllineitems}
\phantomsection\label{api:old.lib.utils.getUnrestrictedUsers}\pysiglinewithargsret{\code{old.lib.utils.}\bfcode{getUnrestrictedUsers}}{}{}
Return the list of unrestricted users in
app\_globals.applicationSettings.applicationSettings.unrestrictedUsers.

\end{fulllineitems}

\index{isLexical() (in module old.lib.utils)}

\begin{fulllineitems}
\phantomsection\label{api:old.lib.utils.isLexical}\pysiglinewithargsret{\code{old.lib.utils.}\bfcode{isLexical}}{\emph{form}}{}
Return True if the input form is lexical, i.e, if neither its morpheme
break nor its morpheme gloss lines contain the space character or any of the
morpheme delimiters.  Note: designed to work on dict representations of forms
also.

\end{fulllineitems}

\index{jsonify() (in module old.lib.utils)}

\begin{fulllineitems}
\phantomsection\label{api:old.lib.utils.jsonify}\pysiglinewithargsret{\code{old.lib.utils.}\bfcode{jsonify}}{\emph{func}}{}
Action decorator that formats output for JSON

Given a function that will return content, this decorator will turn
the result into JSON, with a content-type of `application/json' and
output it.

Adapted from pylons.decorators.

\end{fulllineitems}

\index{makeDirectorySafely() (in module old.lib.utils)}

\begin{fulllineitems}
\phantomsection\label{api:old.lib.utils.makeDirectorySafely}\pysiglinewithargsret{\code{old.lib.utils.}\bfcode{makeDirectorySafely}}{\emph{path}}{}
Create a directory and avoid race conditions.  Taken from 
\href{http://stackoverflow.com/questions/273192/python-best-way-to-create-directory-if-it-doesnt-exist-for-file-write}{http://stackoverflow.com/questions/273192/python-best-way-to-create-directory-if-it-doesnt-exist-for-file-write}.
Listed as make\_sure\_path\_exists.

\end{fulllineitems}

\index{normalize() (in module old.lib.utils)}

\begin{fulllineitems}
\phantomsection\label{api:old.lib.utils.normalize}\pysiglinewithargsret{\code{old.lib.utils.}\bfcode{normalize}}{\emph{unistr}}{}
Return a unistr using canonical decompositional normalization (NFD).

\end{fulllineitems}

\index{normalizeDict() (in module old.lib.utils)}

\begin{fulllineitems}
\phantomsection\label{api:old.lib.utils.normalizeDict}\pysiglinewithargsret{\code{old.lib.utils.}\bfcode{normalizeDict}}{\emph{dict\_}}{}
NFD normalize all unicode values in {\color{red}\bfseries{}dict\_}.

\end{fulllineitems}

\index{removeAllWhiteSpace() (in module old.lib.utils)}

\begin{fulllineitems}
\phantomsection\label{api:old.lib.utils.removeAllWhiteSpace}\pysiglinewithargsret{\code{old.lib.utils.}\bfcode{removeAllWhiteSpace}}{\emph{string}}{}
Remove all spaces, newlines and tabs.

\end{fulllineitems}

\index{restrict() (in module old.lib.utils)}

\begin{fulllineitems}
\phantomsection\label{api:old.lib.utils.restrict}\pysiglinewithargsret{\code{old.lib.utils.}\bfcode{restrict}}{\emph{*methods}}{}
Restricts access to the function depending on HTTP method

Just like pylons.decorators.rest.restrict except it returns JSON.

\end{fulllineitems}

\index{secureFilename() (in module old.lib.utils)}

\begin{fulllineitems}
\phantomsection\label{api:old.lib.utils.secureFilename}\pysiglinewithargsret{\code{old.lib.utils.}\bfcode{secureFilename}}{\emph{path}}{}
Removes null bytes, path.sep and path.altsep from a path.
From \href{http://lucumr.pocoo.org/2010/12/24/common-mistakes-as-web-developer/}{http://lucumr.pocoo.org/2010/12/24/common-mistakes-as-web-developer/}

\end{fulllineitems}

\index{sendPasswordResetEmailTo() (in module old.lib.utils)}

\begin{fulllineitems}
\phantomsection\label{api:old.lib.utils.sendPasswordResetEmailTo}\pysiglinewithargsret{\code{old.lib.utils.}\bfcode{sendPasswordResetEmailTo}}{\emph{user}, \emph{newPassword}, \emph{**kwargs}}{}
Send the ``password reset'' email to the user.  {\color{red}\bfseries{}**}kwargs should contain a
config object (with `config' as key) or a config file name (e.g.,
`production.ini' with `configFilename' as key).  If
password\_reset\_smtp\_server is set to smtp.gmail.com in the config file, then
the email will be sent using smtp.gmail.com and the system will expect a
gmail.ini file with valid gmail\_from\_address and gmail\_from\_password values.
If the config file is test.ini and there is a test\_email\_to value, then that
value will be the target of the email -- this allows testers to verify that
an email is in fact being received.

\end{fulllineitems}

\index{toSingleSpace() (in module old.lib.utils)}

\begin{fulllineitems}
\phantomsection\label{api:old.lib.utils.toSingleSpace}\pysiglinewithargsret{\code{old.lib.utils.}\bfcode{toSingleSpace}}{\emph{string}}{}
Remove leading and trailing whitespace and replace newlines, tabs and
sequences of 2 or more space to one space.

\end{fulllineitems}

\index{userIsAuthorizedToAccessModel() (in module old.lib.utils)}

\begin{fulllineitems}
\phantomsection\label{api:old.lib.utils.userIsAuthorizedToAccessModel}\pysiglinewithargsret{\code{old.lib.utils.}\bfcode{userIsAuthorizedToAccessModel}}{\emph{user}, \emph{modelObject}, \emph{unrestrictedUsers}}{}
Return True if the user is authorized to access the model object.  Models
tagged with the `restricted' tag are only accessible to administrators, their
enterers and unrestricted users.

\end{fulllineitems}

\index{userIsUnrestricted() (in module old.lib.utils)}

\begin{fulllineitems}
\phantomsection\label{api:old.lib.utils.userIsUnrestricted}\pysiglinewithargsret{\code{old.lib.utils.}\bfcode{userIsUnrestricted}}{\emph{user}, \emph{unrestrictedUsers}}{}
Return True if the user is an administrator, unrestricted or there is no
restricted tag.

\end{fulllineitems}



\chapter{Indices and tables}
\label{index:indices-and-tables}\begin{itemize}
\item {} 
\emph{genindex}

\item {} 
\emph{modindex}

\item {} 
\emph{search}

\end{itemize}


\renewcommand{\indexname}{Python Module Index}
\begin{theindex}
\def\bigletter#1{{\Large\sffamily#1}\nopagebreak\vspace{1mm}}
\bigletter{o}
\item {\texttt{old.lib.utils}}, \pageref{api:module-old.lib.utils}
\indexspace
\bigletter{s}
\item {\texttt{simplesite}}, \pageref{api:module-simplesite}
\item {\texttt{simplesite.controllers}}, \pageref{api:module-simplesite.controllers}
\end{theindex}

\renewcommand{\indexname}{Index}
\printindex
\end{document}
